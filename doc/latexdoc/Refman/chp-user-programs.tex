\chapter{User Programs}
\label{chp:user-programs}
\label{bkm:Userprograms}Users have a wide variety of Unix programs which may be used to submit batch jobs, and manage scheduling. This includes
a set of standard command line and interactive programs, plus optional Motif GUI applications.

The following are the user programs available, listed by function, including some intended only for set-up and installation. Some of the
descriptions which follow are merged together to save repetition.

More detailed descriptions of the interactive interfaces to \PrBtq{}, \PrBtuser, \PrXbtq{}, \PrXbtr{}, \PrXbtuser{}, \PrXmbtq{},
\PrXmbtr{} and \PrXmbtuser{} follow in the next two chapters, the descriptions here concentrating on the
command line options to these programs.

\section{Syntax of batch commands}
All of the options referred to in the descriptions of the shell-level programs for \ProductName{} below may be supplied
in a \emphasis{configuration file} (q.v.), or in an environment variable whose name is the same as the calling program,
except that it is in upper case\IfGNU{, with hyphens converted to underscores, thus for example the environment variable name for
passing options to \PrBtq{} is \filename{\BtqVarname}.

Note that the environment variable name is constructed each time from the program name, so if \PrBtq{} is renamed \progname{view-queued-jobs}
then the corresponding environment variable looked for will be \linebreak[20]\exampletext{VIEW\_QUEUED\_JOBS}.}

\IfXi{Note that the environment variable name stays the same even if the program is renamed,
so for example if \PrBtr{} is renamed to be \progname{Queue-a-job}, the environment variable name does not change from being \filename{BTR}.}

This may enable defaults to be supplied according to the application from which the program is invoked. However any options and arguments
supplied on the command line usually take priority.

Additionally by editing the appropriate \emphasis{message file} (q.v.) it is possible to change the option letters and keywords from
those described.

\subsection{Option types}
In nearly all cases there are two alternative ways of supplying options:

\begin{itemize}
\item Via a traditional Unix-style \genericargs{-letter} option, for example as \exampletext{-z}. In some cases, such as in \PrBtr{} and
\PrBtjchange{} we ran out of letters and had to use other a few other characters, such as digits.
\item Via a keyword-style option, such as \exampletext{+zero-charge}. Keywords are case-insensitive.
\end{itemize}
\subsection{Option syntax}
The syntax of options is intended to be as flexible as possible. Options which do not take arguments may be grouped together as in

\begin{expara}

{}-Nwm

\end{expara}

or they may be given separately as in

\begin{expara}

{}-N -w -m

\end{expara}

White space is optional in the case of options which do take arguments, thus both

\begin{expara}

{}-p150

\end{expara}

and

\begin{expara}

{}-p 150

\end{expara}

are acceptable and have the same effect.

If the keyword version of an option is given, then it must be separated from its argument by white space thus

\begin{expara}

+priority 150

\end{expara}

\subsection{Configuration files}
To save the user from having to specify commonly-used combinations of options, there are mechanisms enabling these to be supplied to each
program automatically.

One mechanism is the use of a \emphasis{configuration file}, \configurationfile{},
in the current or a similar file \linebreak[20]\homeconfigpath{} off the user's home directory. The other is the use of an environment variable.

These files may also be used to specify alternative \emphasis{message files}.

The format of configuration files is akin to a set of environment variable assignments, with empty lines and lines beginning with
\exampletext{\#} being ignored.

The name assigned either in the configuration file or that of the environment variable,
to is the same as that of the calling program but in upper case\IfGNU{ and with hyphens replaced by underscores},
for example that corresponding to \PrBtr{} is \filename{\BtrVarname} etc. This is the same as for the corresponding environment variable.

Note that if the program is renamed, for example \PrBtr{} is renamed as \progname{my-queue-program}, then the name of the variable \IfXi{does not
change with it. It still remains as \filename{\BtrVarname}}\IfGNU{changes with it, in that instance to \filename{MY\_QUEUE\_PROGRAM}}.

Usually options are taken from the following places in order, so that later-processed ones override earlier ones:

\begin{tabular}{l p{12cm}}
Standard defaults &
Each program has a set of standard defaults which are used to
initialise the parmaters when the program is invoked.\\
User profile & In some cases, for example default priority, the user's profile as displayed by \PrBtuser{} is used to initialise the defaults.\\
Home options directory & The file \homeconfigpath{} is read, and any options specified therein (i.e. with an assignment to the appropriate
name) are interpreted.\\
Home directory & The file \configurationfile{} in the user's home directory is read, and any options
specified therein (i.e. with an assignment to the appropriate name) are interpreted. This is for compatibility with previous versions of
\ProductName{} and the version using \homeconfigpath{} should be used in new applications to avoid \configurationfile{} being read twice when
commands are run from the user's home directory. \\
Environment &
Any options specified in the appropriate environment variable
(you will almost certainly have to use quotes when setting it via the
shell in order to preserve the white space) are read and interpreted.\\
Current Directory &
The file \configurationfile{} is read, and any options
specified therein (i.e. with an assignment to the appropriate name) are
interpreted.\footnotemark{}. \\
Command line &
Any options specified on the command line are interpreted last.\\
\end{tabular}

\footnotetext{Note that the problem of reading configuration files twice
if programs are run from the home directory has been overcome by moving
the ``home directory'' version from \configurationfile{} to \homeconfigpath{} off the user's home directory.}

Most options have inverses so that it is possible to reset anything which may have been set by previously-read options. Extra care should
be taken with cumulative options, notably arguments and redirections, so that these are not doubled, especially in the case where the home
directory is also the current directory.

\subsection{Option path}
The above description of the order of selection of configuration files, environment and command-line options is the default.

It may be desirable to change the order of selection of options, in to eliminate some alternative locations or to include others.

The environment variable \configpathvar{} may be set to a colon-separated list of directories (environment variables and
\exampletext{\~{}user} constructs are appropriately interpreted).

The symbol \exampletext{!} is used to represent the relevant environment variable, and \exampletext{{}-} is used to
represent option arguments.

The symbole \exampletext{@} is used to represent the ``home directory'' configuration file \homeconfigpath{} for the current user.

The default value of \configpathvar{} is

\begin{expara}

@:\~:!:.:-

\end{expara}

This provides the interpretation of options in various configuration
files and the environment which is documented above.

Note that it is possible to eliminate or override the interpretation of options on the command line by removing or relocating the
\exampletext{{}-}. This may have very surprising effects especially where configuration files wipe out the effects of options
which may have been set on the command line. Where the interpretation of options has been removed altogether, then any options supplied will
probably be objected to or misinterpreted as file names or similar.

The options to most programs of

\begin{expara}

+freeze-home

\end{expara}

and

\begin{expara}

+freeze-current

\end{expara}

and equivalents do not take into account the value of \configpathvar{} in any way.

Please note that \configpathvar{}, with its default and interpretation is the same in \IfXi{other Xi Software products, such as}
\OtherProductName.

Finally please note that any non-existent or inaccessible directories and files will (usually) be silently ignored. If a configuration file
appears to exist but is inaccessible, a diagnostic may be given; however in some cases this may be misleading due to the fact that
various versions of Unix are misleading or inconsistent with regard to the error codes reported from an attempt to open a non-existent or
inaccessible file in a non-existent or inaccessible directory.
\subsection{Message files}
As well as providing help and error messages, screen key assignments etc, message files also provide the option letters and keyword names
used to specify the options.

For each command, there is a default message file. For most of the shell-based commands, this is \filename{btrest.help} in \helpdir.
Alternative message files may be specified using an environment variable or configuration file assigning values to a name. For most of the shell-based commands, this is \filename{BTRESTCONF}.

Within the message file itself, the option letters and keywords are set up using sequences of the form

\begin{expara}

A300:?,explain

\end{expara}

Comments and the context should make clear which commands these options relate to.

These sequences define

\begin{tabular}{lp{12cm}}
A state number & The state number, in the above example \exampletext{300}, which is used internally to denote the argument.\\
& \\
option letters & A single character, often a letter, but in the above example \exampletext{?}, which is the single-character variant of the
option, thus \exampletext{{}-?}.\newline
Several option letters, each separated by commas may be defined. To define ``\exampletext{,}'' itself as
an option ``letter'', use \exampletext{{\textbackslash},}.\footnotemark{}\\
& \\
option keywords & A string of alphanumerics, possibly including hyphens and underscores, is used to denote an option keyword, in the above example
\exampletext{+explain}. Several such keywords may be defined, each separated by commas. Note that the case of letters in the keywords
is discarded.\\
\end{tabular}

\footnotetext{We intend removing this facility to respecify option letters (as opposed to keywords) in future versions of \ProductName{} as it
over-complicates this and gives rise to too many potential conflicts. Please advise us if you think this is a mistake.}
\subsection{Location of message files} 
It is possible to specify alternative locations for message files so that alternatives are selected according to the application being run etc.

The location may be specified using configuration files in a similar fashion to the search for options, except that the search runs the other way.

The search is in the following order:

\begin{tabular}{lp{12cm}}
Current Directory &
If a configuration file \configurationfile{} in the current directory specifies a
location for the message file, by means of an assignment to the
relevant variable (for most shell-based commands this is
\filename{BTRESTCONF}), then this file is taken.

\bigskip

Environment variables in the form \exampletext{\$ABC} and users{\textquotesingle} home directories in the form \exampletext{\~{}user} are
appropriately expanded. The sequence \exampletext{\$0} is replaced by the name of the program being invoked. (This process may
run recursively up to a level of 10).\\
& \\
Environment &
If the relevant environment variable (for most shell-based commands this is \filename{BTRESTCONF}) specifies a location,
then this is taken.\\
& \\
Home Directory config file &
A configuration file \configurationfile{} in the home directory may specify a location for the message file. (Note that this is deprecated in favour
of using \homeconfigpath{} instead but is still supported for compatibility).\\
& \\
Home Directory &
A configuration file \homeconfigpath{} off the user's home directory may specify a location for the message file.\\
& \\
Default Location & If none of the above specify a replacement message file then the default location is taken.\\
\end{tabular}

If a file is specified but does not exist, then this is a fatal error.

However there is an additional step to assist the user to set up some alternative files with a default name.

Should the file not exist, then the search falls back to a name generated by taking the last part of the default file name (for example
\exampletext{btrest.conf}) and substituting this for the last part of the file name specified.

For example if the normal message file for the command were

\begin{expara}

\helpdirname/btrest.help

\end{expara}

and the user had specified in a configuration file

\begin{expara}

BTRESTCONF=\~{}/\$0.help

\end{expara}

then if he or she were to run, say, \PrBtr{}, then the file

\begin{expara}

\~{}/\BtrName.help

\end{expara}

would be searched for. If this did not exist, then a search would be made for

\begin{expara}

\~{}/btrest.help

\end{expara}

\subsection{Path to locate message files}
The above search path may be modified using the environment variable \helppathvar. The interpretation is very similar to
the description above for \configpathvar, except that \exampletext{{}-} fields are ignored.

The default value of \helppathvar{} is \filename{.:!:\~:@} giving the interpretation described above. Note that this is in the opposite order to
\configpathvar.

\section{Submitting Batch Jobs}
\subsection{\BtrName{} and \RbtrName{}}

\begin{expara}

\BtrName{} [-options] [ files ]

\RbtrName{} [-options] [ files ]

\end{expara}

\PrBtr{} creates a \ProductName{} batch job from each of the supplied files or the standard input if no file names are given.

\PrRbtr{} operates similarly, but creates the jobs on a remote host without the necessity of having to have \ProductName{} running on the submitting host.

\subsubsection{Options}
Except for the \exampletext{{}-Q} option, which must be specified for \PrRbtr{}, and the options keyword used
to pick up default arguments and to save with the \exampletext{+freeze-current} and \exampletext{+freeze-home} options, the options to \PrRbtr{} are identical in effect to those for \PrBtr{}\footnote{Standard - it would be possible to make them different by editing the option definitions in
\filename{\helpdirname/btrest.help} but this would not be sensible}.

The environment variable on which options are supplied is
\filename{\BtrVarname} for \PrBtr{}, \filename{\RbtrVarname} for \PrRbtr{} and the environment variable to specify the help file is
\filename{BTRESTCONF}.

We regret having run out of single letters for options to \PrBtr{} and \PrRbtr{} and having had to resort in three cases to non-alphabetic options. The
next release of \ProductName{} introduces the new concept of \textit{templates} carrying most of the information supplied as options to
\PrBtr{}.
\setbkmkprefix{btr}
\explainopt

\cmdoption{2}{grace-time}{time}{gracetime}

This option sets the second stage time of handling over-running jobs to time, in seconds (the argument may be any number of seconds, or given
as \genericargs{mm:ss} for minutes and seconds).

This only applies if a maximum elapsed time for a job is set with the \exampletext{{}-Y} option. If a non-zero time is also given
with this option, the job is first killed with the signal number given by the \exampletext{{}-W} option and then, if it continues to
run for the time given by this argument, killed with \filename{SIGKILL} which cannot be caught or ignored.

\cmdoption{9}{catch-up}{~}{catchup}

This option sets the ``if not possible'' action of the job or jobs to catch up - one run of a series of missed runs is done when it is possible without affecting future runs.

\cmdoption{.}{done}{~}{done}

This option sets the job or jobs to ``done'' state.

This is mainly intended for resubmitting jobs which have been ``unqueued'' and is not recommended for general use.

\cmdoption{A}{avoiding-days}{~}{avoiding}

This option specifies days to avoid when the job or jobs are to be repeated automatically. The days to avoid option supersedes any
preceding or default option, unless a leading comma is given.

Thus if the existing days to avoid are \exampletext{Sat} and \exampletext{Sun}, the default when installed,

\begin{expara}

\BtrName{} -A Wed ...

\end{expara}

will change the days to avoid to be Wednesday only, whereas

\begin{expara}

\BtrName{} -A ,Wed ...

\end{expara}

will change the days to avoid to be Saturday, Sunday and Wednesday.

A single \exampletext{{}-} argument cancels the days to avoid parameter altogether, thus

\begin{expara}

\BtrName{} -A {}- ...

\end{expara}

Note that this parameter only affects automatic repetitions, so if the date given by the \exampletext{{}-T} parameter falls on a day
excluded by this argument, it will not be affected and the first run will be on the date specified.

Upon installation the default abbreviations for the days are \exampletext{Sun}, \exampletext{Mon}, \exampletext{Tue}, \exampletext{Wed},
\exampletext{Thu}, \exampletext{Fri}, \exampletext{Sat} and \exampletext{Hday}, the last refers to holidays as specified in the holiday file.
The days are interpreted case-insensitively, but on saving options with \exampletext{+freeze-current} or
\exampletext{+freeze-home} will save the names as given in the message file, by default in the initial capital format.

\cmdoption{a}{argument}{string}{arg}

Provide an argument string to the command interpreter. Successive \exampletext{{}-a} options are cumulative and
append additional arguments to the list of arguments for the job or jobs. To clear previously-specified options (maybe set in
\configurationfile{} files) and start afresh, use the \exampletext{{}-e} (see page \pageref{btr:cancarg}) option first.

\cmdoption{B}{assignment-not-critical}{~}{anotcrit}

This marks subsequently-specified assignments (with the \exampletext{{}-s} option) as ``not critical'', meaning that the assignment will be ignored if
it contains a reference to a variable on a remote host which is offline or inaccessible.

This must precede (not necessarily immediately) the \exampletext{{}-s} options to which it is to be applied.

\cmdoption{b}{assignment-critical}{~}{acrit}

This marks subsequently-specified assignments (with the \exampletext{{}-s} option) as ``critical'', meaning that the job or jobs
will not start if the assignment contains a reference to a variable on a remote host which is offline or inaccessible.

This must precede (not necessarily immediately) the \exampletext{{}-s} options to which it is to be applied.

\cmdoption{C}{cancelled}{~}{cancelled}

This causes the job or jobs to be queued in the ``cancelled'' state.

This is commonly used to initially queue a job for later editing by a GUI-style utility such as \PrBtq{}, \PrXbtq{} and \PrXmbtq{}.

\cmdoption{c}{condition}{condition}{condition}

This sets a condition to be satisfied before the job or jobs may run.

Successive \exampletext{{}-c} options cause further conditions to be appended to the list, up to a maximum of 10 conditions.

To start from scratch, deleting any previously-specified conditions (in a \configurationfile{} file perhaps), use the
\exampletext{{}-y} option first.

The format of the condition argument is decribed fully on page \pageref{btr:condfmt}.

\cmdoption{D}{directory}{directory}{directory}

This option sets the working directory for the job or jobs.

The directory may be specified using environment variables preceded by \exampletext{\$} or constructs to denote a user's home directory
of the form \genericargs{\~{}user} such as in

\begin{expara}

\$HOME/batchjobs

\~jim/jobs

\end{expara}

Remember, if using the shell, and using these constructs, to put quotes around the directory, otherwise the shell may expand the constructs and not
\ProductName{}). This may well have the intended effect in most cases, but for jobs to be ``portable'' across different
hosts for remote execution, it is better for the expansions to be done as late as possible.

If this option is omitted, then the current directory at the time of invoking \PrBtr{} or \PrRbtr{} is used.

\cmdoption{d}{delete-at-end}{~}{deleteatend}

This option cancels any repeat option of the job or jobs so that they will be deleted at the end of the run rather than repeated or kept. This is the
default if no arguments are specified.

\cmdoption{E}{local-environment}{~}{locenv}

This option only applies to \PrRbtr{} and is ignored by \PrBtr{}.

It instructs \PrRbtr{} to use the environment variables from the local environment (from which the job is submitted)
rather than the remote environment (to which the job is going).

\cmdoption{e}{cancel-arguments}{~}{cancarg}

This option cancels any arguments previously set up with the \exampletext{{}-a} option.

You might want to use it if your environment or a \configurationfile{} file specifies standard arguments and you want to clear those and start again.

\cmdoption{F}{export}{~}{export}

This marks the job or jobs to be visible throughout the network or ``exported''.

The job won't actually run on another host unless you go further and make the job ``remote runnable'' with the
\exampletext{{}-G} option, as described on page \pageref{btr:fullexport}.

\cmdoption{f}{flags-for-set}{letters}{assflags}

This option provides a set of ``flags'' for subsequent assignment operators specified by the \exampletext{{}-s} option, indicating when they
should apply.

The argument letters should be some or all of \exampletext{SNEACR} for respectively Start, Normal exit, Error exit, Abort, Cancel and Reverse.

\cmdoption{G}{full-export}{~}{fullexport}

This option marks the job or jobs to be visible throughout the network and potentially available to run on any machine.

\cmdoption{g}{set-group}{~}{group}

This option sets the group owner of the job or jobs to be \genericargs{group}. The user must have ``write admin file'' permission to invoke this option.

\cmdoption{H}{hold-current}{~}{holdcurrent}

This option selects the variant of the ``if not possible'' action for the job or jobs to ``hold current''. The next run is done
when possible, but the usual time is not adjusted.

Note that unlike with the ``catch up'' option described on page \pageref{btr:catchup},
subsequent runs are not omitted, the job will repeatedly run until all missed runs are completed.

\cmdoption{h}{title}{title}{title}

This option supplies a title for the job or jobs, setting it to the supplied \genericargs{title}.

In the absence of this argument the title will be that of the last part of the file name, if any.

The title may be a string of any length containing any printable characters, but colon should be
avoided to avoid confusion with queue names.

If the title contains spaces or characters interpreted by the shell, it should be surrounded by
quotes.

\cmdoption{I}{input-output}{redirection-spec}{redir}

This option specifies a redirection for the job or jobs. Successive \exampletext{{}-I} options are cumulative and will append to
the current list of redirections. To start the list of redirections from scratch, precede them with the \exampletext{{}-Z}
option.

When the job is executed the redirections are handled in order from first to last.

The format of redirection specifications are described fully on page \pageref{btr:redirfmt}.

\cmdoption{i}{interpreter}{name}{interpreter}

This option sets the command interpreter for the job or jobs to be that specified by the name,
which should already be defined.

The load level is also set to that for the specified interpreter, so if a
\exampletext{{}-l} argument is to be specified, it should \emphasis{follow} the \exampletext{{}-i} option.

\cmdoption{J}{no-advance-time-error}{~}{noadv}

This specifies that if the job exits with an error, the next time to run is not advanced according to the repeat specification if applicable.

\cmdoption{j}{advance-time-error}{~}{adv}

This specifies that if the job exits with an error, the next time to run is still advanced if applicable according to the repeat specification.

This is the default if no arguments are specified.

\cmdoption{K}{condition-not-critical}{~}{cnotcrit}

This option marks subsequently specified conditions set with the \exampletext{{}-c} option as ``not critical'', i.e. a condition dependent on a variable on an offline or otherwise inaccessible remote host will be ignored in deciding whether a job may start.

This must precede (not necessarily immediately) the \exampletext{{}-c} options to which it is to be applied.

This setting is the default if no arguments are specified.

\cmdoption{k}{condition-critical}{~}{ccrit}

This option marks subsequently specified conditions set with the \exampletext{{}-c} option as ``critical'', i.e. a condition dependent on a
variable on an offline or otherwise inaccessible remote host will cause the job to be held up.

This must precede (not necessarily immediately) the \exampletext{{}-c} options to which it is to be applied.

\cmdoption{L}{ulimit}{value}{ulimit}

This option sets the \filename{ulimit} (maximum file size) value for the job or jobs to the value given.

Set a value of zero (the default) to indicate an unlimited value.

We strongly recommend that this option not be used as it easily causes a lot of unexpected problems.

\cmdoption{l}{loadlev}{number}{loadlev}

This option sets the load level of the job or jobs to be number. The user must have ``special create permission'' for this to differ from that of the
command interpreter.

The load level is also reset by the \exampletext{{}-i} (set command interpreter) option, so this option must be used after that has been specified if it is to have any effect.

\cmdoption{M}{mode}{modes}{modes}

This option sets the permission of the job or jobs to be as specified.

The format of the mode argument is described fully on page \pageref{btr:modefmt}.

\cmdoption{m}{mail-message}{~}{mail}

This option sets the flag whereby completion messages are mailed to the owner of the job. (They may anyway if the jobs output to standard
output or standard error and these are not redirected).

\cmdoption{N}{normal}{~}{normal}

This option sets the job or jobs to normal ``ready to run'' state, as opposed to ``cancelled'' as set by the \exampletext{{}-C} option.

This is the default if no arguments are specified.

\cmdoption{n}{local-only}{~}{localonly}

This option marks the job or jobs to be local only to the machines which they are queued on, cancelling any \exampletext{{}-F} or \exampletext{{}-G}
options. They will not be visible or runnable on remote hosts.

This is the default if no arguments are specified.

\cmdoption{O}{remote-environment}{~}{remenv}

This option only applies to \PrRbtr{} and is ignored by \PrBtr{}.

It instructs \PrRbtr{} to use the environment variables from the remote environment (to which the job is going) rather than the local
environment (from which the job is submitted).

This is the default if no arguments are specified.

\cmdoption{o}{no-repeat}{~}{norep}

This option cancels any repeat option of the job or jobs, so that the they will be run and retained on the queue marked ``done'' at the end.

\cmdoption{P}{umask}{value}{umask}

This option sets the \progname{umask} value of the job or jobs to the octal value given. The value should be up to 3 octal digits as
per the shell.

The initial value, if no other option is given, is taken from the invoking environment.

\cmdoption{p}{priority}{number}{priority}

This option sets the priority of the job or jobs to be \genericargs{number}, which must be in the range given by the user{\textquotesingle}s minimum
and maximum priority.

\cmdoption{Q}{host}{hostname}{host}

This option primarily applies to \PrRbtr{}, for which it is always required, and specifies that the job should
be sent to to the given \textit{hostname}.

If specified with \PrBtr{}, the effect is to invoke \PrRbtr{} with the same command-line options as were given to \PrBtr{}.

Note that this does not include any options for \PrBtr{} extracted from the environment or \configurationfile{} files or the environment using the
\filename{\BtrVarname} keyword, the automatically-invoked \PrRbtr{} will pick up its own options from those files using the \filename{\RbtrVarname} keyword.

There is no loop if the options for \PrRbtr{} don't thereby specify this option, an error message is given instead.

\cmdoption{q}{job-queue}{queuename}{queue}

This option sets a job queue name as specified on the job or jobs. This may be any sequence of printable characters.

Currently the queue name is just a prefix on the job title. In future releases of \ProductName{}, it is much more sophisticated.

\cmdoption{R}{reschedule-all}{~}{delayall}

This option sets the ``not possible'' action of the job or jobs to reschedule all - the run is done when it is possible
and subsequent runs are rescheduled by the amount delayed.

\cmdoption{r}{repeat}{repeat\_spec}{repeat}

This option sets the repeat option of the job or jobs as specified.

The format of the repeat argument is described fully on page \pageref{btr:repeatfmt}.

\cmdoption{S}{skip-if-held}{~}{skip}

This option sets the ``not possible'' action of the job or jobs to skip - the run is skipped if it could not be done at the specified
time.

\cmdoption{s}{set}{assignment}{assign}

This option sets an assignment on the job or jobs to be performed at the start and/or finish of the job or jobs as selected by a
previously-specified \exampletext{{}-f} option (see page \pageref{btr:assflags}).

This option is cumulative, and will add to the list of assignments specified by previous \exampletext{{}-s} options.

To start from scratch, precede the assignments with the \exampletext{{}-z} option.

The format of the assignment argument is described fully on page \pageref{btr:assfmt}.

\cmdoption{T}{time}{time}{time}

This option sets the next run time or time and date of the job or jobs as specified.

\cmdoption{t}{delete-time}{time}{deltime}

This option sets a delete time for the specified job or jobs as a time in hours, after which it will be automatically deleted if this time has
elapsed since it was queued or last ran.

Set to zero (the default) to retain the job or jobs indefinitely.

\cmdoption{U}{no-time}{~}{notime}

This option cancels any time setting on the job or jobs set with \exampletext{{}-T}, \exampletext{{}-r} or \exampletext{{}-o} options.

\cmdoption{u}{set-owner}{user}{owner}

This option sets the owner of the job or jobs to be \genericargs{user}.

The invoking user must have ``write admin file'' permission to use this option.

\cmdoption{V}{no-verbose}{~}{noverbose}

This option suppresses any confirmation message about the successful submission of jobs.

This is the default if no options are given.

\cmdoption{v}{verbose}{~}{verbose}

This option causes \PrBtr{} and \PrRbtr{} to display a message on standard error with the job number of each job successfully created.

\cmdoption{W}{which-signal}{sig}{whichsig}

This option is used in conjunction with the \exampletext{{}-Y} and \exampletext{{}-2} options to specify the initial signal number, e.g.
1, 2, 15 to kill the job or jobs after the maximum run time has been exceeded.

\cmdoption{w}{write-message}{~}{write}

This option is used to indicate that completion messages are written to the owner's terminal if available.

\cmdoption{X}{exit-code}{range}{exits}

This option sets the normal or error exit code ranges for the job or jobs.

The format of the range argument is \exampletext{N} or \exampletext{E} followed by a range in the form \genericargs{nn:nn}, thus

\begin{expara}

{}-X N0:9

{}-X E10:255

\end{expara}

Note that an exit code which falls inside both ranges will be handled by the setting of the \emphasis{smaller} range, so

\begin{expara}

{}-X N0:10

{}-X E1:255

\end{expara}

will mean that exit codes 1 to 10 inclusive are treated as normal as that is the smaller range. Unhandled exit codes are treated
as abort.

The default ranges are \exampletext{N0:0} and \exampletext{E1:255}.

\cmdoption{x}{no-message}{~}{nomess}

This option resets both flags as set by the \exampletext{{}-m} and \exampletext{{}-w} options.

\cmdoption{Y}{run-time}{time}{runtime}

This option sets a maximum elapsed run time for the specified job or jobs.

The argument time is in seconds, which may be written as \genericargs{mm:ss} or \genericargs{hh:mm:ss}.

The job will be killed with \filename{SIGKILL} unless a different signal is specified with the \exampletext{{}-W} option and a further ``grace time'' specified with the \exampletext{{}-2} option.

\cmdoption{y}{cancel-condition}{~}{canccond}

This option deletes any conditions set up by previous \exampletext{{}-c} options.

This may be necessary if the environment or a \configurationfile{} file contain \exampletext{{}-c} options for defaults and it is required to remove those and specify other ones.

\cmdoption{Z}{cancel-io}{~}{cancio}

This option deletes any redirections set up by previous \exampletext{{}-I} options.

This may be necessary if the environment or a \configurationfile{} file contain \exampletext{{}-I}
options for defaults and it is required to remove those and specify other ones.

\cmdoption{z}{cancel-set}{~}{cancass}

This option deletes any assignments set up by previous \exampletext{{}-s} options.

This may be necessary if the environment or a \configurationfile{} file contain
\exampletext{{}-s} options for defaults and it is required to remove those and specify other ones.

\freezeopts{\filename{\BtrVarname} for \PrBtr{} or \filename{\RbtrVarname} for \PrRbtr{}}{Stop}

\subsubsection{Redirection format}
\bookmark{redirfmt}
The format of the argument to the \exampletext{{}-I} option is similar to that for the shell with some extensions. The argument should
always be enclosed in quotes to avoid the shell interpreting it rather than \PrBtr{} or \PrRbtr{}.

Environment variables and \genericargs{\~{}user} constructs are expanded at run time in the strings.

Parameter substitutions, or ``meta data'' may be included in the argument strings for redirections, see meta data on page \pageref{bkm:Metadata}.

\paragraph{Input from file}

The redirection style:

\begin{expara}

\textit{n}{\textless}file

\end{expara}

For example

\begin{expara}

3{\textless}myfile

7{\textless}/tmp/data

{\textless}input\_file

\end{expara}

Opens the specified file descriptor for input connected to the specified file. The file descriptor may be omitted in the common case
of file descriptor 0 (standard input).

\paragraph{Output to file}

The redirection style:

\begin{expara}

\textit{n}{\textgreater}file

\end{expara}

For example

\begin{expara}

4{\textgreater}outfile

2{\textgreater}errors.\%t

12{\textgreater}/tmp/out

{\textgreater}output\_file

\end{expara}

Opens the specified file descriptor for output, possibly creating the file, or truncating it to zero length first if it exists.
The file descriptor may be be omitted in the common case of file descriptor 1 (standard output).

\paragraph{Append to file}

The redirection style:

\begin{expara}

\textit{n}{\textgreater}{\textgreater}file

\end{expara}

For example

\begin{expara}

5{\textgreater}{\textgreater}Log

7{\textgreater}{\textgreater}Log.\%t

{\textgreater}{\textgreater}output.\%t

\end{expara}

As with the shell, this likewise creates the output file if it does not exist but appends new data to any previous data if it exists,
rather than truncating it.

\paragraph{Open for read and write}

The redirection style:

\begin{expara}

\textit{n}{\textless}{\textgreater}file

\end{expara}

For example

\begin{expara}

8{\textless}{\textgreater}Data

{\textless}{\textgreater}Myfile

\end{expara}

Connect the file descriptor (or file descriptor 0 if not specified) for input and output, read-write mode. The file is created if it
does not exist and is truncated if it does exist.

\paragraph{Open for read and append}

The redirection style:

\begin{expara}

\textit{n}{\textless}{\textgreater}{\textgreater}file

\end{expara}

For example

\begin{expara}

8{\textless}{\textgreater}{\textgreater}Data

{\textless}{\textgreater}{\textgreater}Myfile

\end{expara}

Connect the file descriptor (or file descriptor 0 if not specified) for input and output, read-write mode. The file is created if it
does not exist and data is appended to the end of the file if it does exist.

\paragraph{Input from program}

The redirection style:

\begin{expara}

\textit{n}{\textless}{\textbar}program

\end{expara}

For example

\begin{expara}

7{\textless}{\textbar}uname

\end{expara}

Run the specified program and take input from it on the given file descriptor (defaulting to standard input, file descriptor 0, if not specified).

\paragraph{Output to program}

The redirection style:

\begin{expara}

\textit{n}{\textbar}program

\end{expara}

For example

\begin{expara}

2{\textbar}log\_errors

{\textbar}log\_output7{\textless}{\textbar}uname

\end{expara}

Run the specified program and send output to it on the given file descriptor (defaulting to standard output, file descriptor 1, if not specified).

\paragraph{Duplicate descriptor}

The redirection style:

\begin{expara}

\textit{n\&k}

\end{expara}

For example

\begin{expara}

2\&1

\end{expara}

Duplicates the second file descriptor as the first file descriptor with the same attributes.

\paragraph{Close descriptor}

The redirection style:

\begin{expara}

\textit{n\&-}

\end{expara}

Closes the given file descriptor. (Note that redirections are always treated first to last).

\bookmark{repeatfmt}
\subsubsection{Repeat periods}
The repeat period names for the \filename{-r} option are as follows:

\begin{center}
\begin{tabular}{l l}
\exampletext{Minutes} & Period in minutes\\
\exampletext{Hours} & Period in hours\\
\exampletext{Days} & Period in days\\
\exampletext{Weeks} & Period in weeks\\
\exampletext{Monthsb} & Months relative to the beginning\\
\exampletext{Monthse} & Months relative to the end of the month\\
\exampletext{Years} & Period in years\\
\end{tabular}
\end{center}
Each is followed by the number of the relevant periods after a colon. In the case of the month parameters, then this should be followed by a
``target day'' after a colon.

Examples:

\begin{expara}

{}-r Days:4

{}-r Monthsb:1:4

{}-r Monthse:1:31

{}-r Years:2

\end{expara}

For \exampletext{Monthsb} the ``target day'' is the day of the month to aim for, in this case the 4th of the month.
If this would be a ``day to avoid'', then the following day is tried and so on.

For \exampletext{Monthse} the ``target day'' is selected from the day of the month given in the \exampletext{{}-T} option. So if the month in the
\exampletext{{}-T} option has 31 days, then \exampletext{{}-r Monthse:1:31}

will select the last day of each month and

\begin{expara}

{}-r Monthse:1;30

\end{expara}

will select the second last, but if the month in the \exampletext{{}-T} option has 30 days, the first will be invalid and the second will select the last day of the month.

If the selected day cannot be met for any reason, typically because it does not meet the ``days to avoid'' criteria,
then the previous day is tried until an acceptable day is found. In this way you can select the ``last working day of the
month'' or ``next to last working day'' etc.

\subsubsection{Conditions}
\bookmark{condfmt}
A condition must be of the form

\begin{expara}

[machine:]{\textless}varname{\textgreater}{\textless}condop{\textgreater}{\textless}constant{\textgreater}.

\end{expara}

where \exampletext{varname} is the name of an existing variable for which the user has read permission.

\exampletext{condop} is one of the following:

\begin{center}
\begin{tabular}{l l}
\exampletext{=} & equal to\\
\exampletext{!=} & not equal\\
\exampletext{{\textless}} & less than\\
\exampletext{{\textless}=} & less than or equal\\
\exampletext{{\textgreater}} & greater than\\
\exampletext{{\textgreater}=} & greater than or equal\\
\end{tabular}
\end{center}
constant is either a string or a numeric value. If the string starts with a number then it should be preceded with a colon.

N.B. When invoked from a shell, quotation marks should surround the entire argument as shown above, otherwise the shell may attach its own
interpretation on the various characters.

Examples of conditions:

\begin{expara}

{}-c {\textquotesingle}Count{\textgreater}3{\textquotesingle}

{}-c {\textquotesingle}Lock=0{\textquotesingle}

{}-c {\textquotesingle}Remote:Lock!=0{\textquotesingle}

{}-c {\textquotesingle}Val=:3rd{\textquotesingle}

\end{expara}

\subsubsection{Assignments}
\bookmark{assfmt}
Each assignment should normally be preceded by a \exampletext{{}-f} option to denote when the assignment is applied, apart from exit code and signal assignments.

The argument to the \exampletext{{}-f} option is one or more of the following:

\begin{center}
\begin{tabular}{l l}
\exampletext{S} & Perform assignment on startup\\
\exampletext{N} & Perform assignment on normal exit\\
\exampletext{E} & Perform assignment on error exit\\
\exampletext{A} & Perform assignment on abort\\
\exampletext{C} & Perform assignment on cancellation\\
\exampletext{R} & Reverse assignment for \exampletext{N}, \exampletext{E}, \exampletext{A}, and \exampletext{C}.\\
\end{tabular}
\end{center}
The default if no \exampletext{{}-f} options are specified is

\begin{expara}

-f SNEAR

\end{expara}

but the default for this may be changed by editing the message file.

The format of the argument to the \exampletext{{}-s} option is in the format

\begin{expara}

[machine:]{\textless}varname{\textgreater}{\textless}operator{\textgreater}{\textless}constant{\textgreater}

\end{expara}

\exampletext{varname} is the name of a variable to which the user has read and write permission.

\exampletext{operator} is one of the following:

\begin{tabular}{lp{12cm}}
\exampletext{=} &
Assign value which may be a string or numeric constant. To indicate that a string starting with a digit is intended to be a
string, prefix it with a colon. Exceptionally, the variable assigned to may have write permission and not read permission for the user.
The effect of the ``reverse'' flag is to assign zero or the null string. Previous values are not recalled.\\
& \\
\exampletext{+=} &
Increment variable by numeric constant. The effect of the ``reverse'' flag is to decrement the variable
by that constant. Arithmetic is as 32-bit signed integer.\\
& \\
\exampletext{{}-=} &
Decrement variable by numeric constant. The effect of the ``reverse'' flag is to increment the variable
by that constant.
Arithmetic is as 32-bit signed integer.\\
& \\
\exampletext{*=} &
Multiply variable by numeric constant. The effect of the ``reverse'' flag is to divide the variable by that constant.
Arithmetic is as 32-bit signed integer and overflow is ignored.\\
& \\
\exampletext{/=} &
Divide variable by numeric constant. The effect of the ``reverse'' flag is to multiply the variable by that constant.
Arithmetic is as 32-bit signed integer. Note that the remainder from division is ignored.
The handling of negative numbers may be dependent on the hardware and
should probably not be relied upon.\\
& \\
\exampletext{\%=} &
Take the remainder (modulus) from division by the numeric constant. There is no ``reverse'' of the operation.
Arithmetic is as 32-bit signed integer. The handling of negative numbers may be dependent on the hardware and should probably not be relied
upon.\\
& \\
\exampletext{=exitcode} &
Assign the exit code of the job to the given variable. Flags are ignored and the operation only occurs when the job exits.\\
& \\
\exampletext{=signal} &
Assign the signal number with which the job terminated to the given variable, or zero if the job did not exit with a signal. Flags
are ignored and the operation only occurs when the job exits.\\
\end{tabular}

The following are examples of assignments:

\begin{expara}

{}-s {\textquotesingle}myvar=7{\textquotesingle}

{}-s {\textquotesingle}host2:hisvar+=1{\textquotesingle}

{}-s {\textquotesingle}status=exitcode{\textquotesingle}

{}-s {\textquotesingle}val=:3rd{\textquotesingle}

\end{expara}

Note the colon in the last assignment indicating that the value is a string, the colon is not included in the string.

\subsubsection{Mode arguments}
\bookmark{modefmt}
The argument to the \exampletext{{}-M} option provides for a wide variety of operations.

Each permission is represented by a letter, as follows:

\begin{center}
\begin{tabular}{l l}
\exampletext{R} & read permission\\
\exampletext{W} & write permission\\
\exampletext{S} & reveal permission\\
\exampletext{M} & read mode\\
\exampletext{P} & set mode\\
\exampletext{U} & give away owner\\
\exampletext{V} & assume owner\\
\exampletext{G} & give away group\\
\exampletext{H} & assume group\\
\exampletext{D} & delete\\
\exampletext{K} & kill\\
\end{tabular}
\end{center}
Each section of the mode (user, group, others) is represented by the prefixes \exampletext{U:}, \exampletext{G:} and
\exampletext{O:} and separated by commas.

For example:

\begin{expara}

{}-M U:RWSMPDK,G:RWSDK,O:RS

\end{expara}

would set the permissions for the user, group and others as given. If the prefixes are omitted, as in

\begin{expara}

{}-M RWSDK

\end{expara}

then all of the user, group and other permissions are set to the same value.


\subsection{\BtsName}

\begin{expara}

\BtsName{} [-options] [ files ]

\end{expara}

\PrBts{} creates and submits a \ProductName{} batch job from each of the supplied XML files.

XML format job files, usually with suffix \batchjobsuffix, are created when jobs are unqueued with the XML option with \PrBtq{} and similar.

\subsubsection{Options}
The environment variable on which options are supplied is \filename{\BtsVarname} and the environment variable to specify the help file is
\filename{BTRESTCONF}.
\setbkmkprefix{bts}
\explainopt

\cmdoption{v}{verbose}{~}{verbose}

This option causes a message showing the job number of each successfully submitted job to be displayed, whether or not the job file
was created with the verbose option.

\cmdoption{q}{quiet}{~}{quiet}

This option turns off the verbose option for each job submitted. No output is displayed for any job.

\cmdoption{f}{verbose-as-file}{~}{verbasfile}

This option resets the ``verbose'' option so that messages are only displayed if the verbose option is given in the file.

\cmdoption{Q}{host}{host}{host}

This option selects the host to which jobs are to be queued, other than from the host on which they were unqueued from. Specify a single ``-'' to
denote the local host.

\cmdoption{F}{host-as-file}{~}{hostasfile}

Reset the \exampletext{{}-Q} option, so that jobs are queued to the host they originated from.

\cmdoption{C}{cancelled}{~}{cancelled}

This causes the job or jobs to be queued in the ``cancelled'' state rather than whatever state they are set to in the job files.

\cmdoption{N}{normal}{~}{normal}

This causes the job or jobs to be queued in the ``ready to run'' state rather than whatever state they are set to in the job files.

\cmdoption{S}{state-as-file}{~}{stateasfile}

This option resets the \exampletext{{}-C} or \exampletext{{}-N} options so that the jobs are queued in the state that they were set
to in the saved job files.

This is the default if no arguments are specified.

\freezeopts{\filename{\BtsVarname}}{Stop}

\subsubsection{Error Messages}

Appropriate error messages are displayed if the saved job files do not appear to be in XML format, or if \ProductName{} was not built
with the XML library.

Note that job files do not have to have the suffix \batchjobsuffix.



\subsection{atcover}

\begin{expara}

atcover options

\end{expara}

\progname{atcover} may be used instead of the standard \progname{at(1)} command. It converts the options that most versions of the
\progname{at(1)} commands take to the equivalent \PrBtr{} commands and then invokes \PrBtr{} to submit the batch jobs.

The \PrBtr{} program provides a much greater set of facilities than \progname{at}(1). Although it is strongly
recommended to switch to the \PrBtr{} command to take advantage of these, they can be made available to users of our
\progname{atcover} command. This is done by setting up \filename{BTR} in the application or user environment.

\progname{atcover} is usually installed in place of \progname{at} in \filename{/usr/bin}, with the
original binary moved to something like \filename{old.at} in the same directory.


\subsection{\XbtrName{} and \XmbtrName}

\begin{expara}

\XbtrName{} \&

\XmbtrName{} \&

\end{expara}

\PrXbtr{} is a fully interactive GTK+ alternative and \PrXmbtr{} is a fully interactive Motif alternative to
the standard tools for submitting batch jobs, \PrBtr{} and \PrRbtr{}. Jobs are submitted from saved job files, which may have been created via
``unqueue'' from \PrBtq{}, \PrXbtq{}, \PrXmbtq{} or \PrBtjdel{}, or created afresh within \PrXbtr{} or \PrXmbtr{}.

Unlike \PrBtr{} etc there are no specific command line options to \PrXbtr{} or \PrXmbtr{}. The facility to change or specify resources settings
for an X11 (and hence GTK+ or Motif) program on the command line can be used.

Because they are set-user, you may need to turn off X11 restrictions by running:

\begin{expara}

xhost +

\end{expara}

before using \PrXbtr{} or \PrXmbtr.


\section{Managing the batch scheduler}
\subsection{\BtstartName}

\begin{expara}

\BtstartName{} [-options]

\end{expara}

\PrBtstart{} initiates the \ProductName{} batch scheduler system, by starting the processes \progname{btsched} and
\progname{xbnetserv}.

\subsubsection[Options]{Options}
The environment variable on which options are supplied is \filename{\BtstartVarname} and the environment variable to specify the
help file is \filename{BTRESTCONF}.

\PrBtstart{} does not do anything, and most of the options are obviously ignored, if the scheduler is already running.

\setbkmkprefix{btstart}
\explainopt

\cmdoption{l}{initial-load-level}{number}{initll}

This option arranges for the \filename{LOADLEVEL} variable, which controls the total load level of running jobs to the
specified number (usually zero).

This is useful for starting up in a controlled fashion, checking the status of jobs and then resetting \filename{LOADLEVEL} appropriately allowing jobs to run.

If this option is not specified, then the value is unchanged from its initial value saved by the scheduler when it was last shut down.

\cmdoption{j}{initial-job-size}{number}{initjobs}

Allocate space for the the specified maximum number of jobs on startup.

This is very often necessary as it may not be possible to reallocate additional shared memory after processing has got under way.

If this option is not specified, then the initial allocation will be based on the original number of saved jobs, which is often far from enough.

\cmdoption{v}{initial-var-size}{number}{initvars}

Allocate space for the specified maximum number of variables on startup.

This is very often necessary as it may not be possible to reallocate additional shared memory after processing has got under way.

If this option is not specified, then the initial allocation will be based on the original number of saved variables, which is often far from enough.

\freezeopts{\filename{\BtstartVarname}}{Stop}


\subsection{\BtquitName}

\begin{expara}

\BtquitName{} [-y]

\end{expara}

\PrBtquit{} is a program which should be invoked to bring the \ProductName{}
batch system to an orderly halt prior to system shutdown. Any jobs and variables will be saved.

Only a user with the stop scheduler privilege may successfully invoke it, and confirmation is requested unless the
\exampletext{{}-y} option is given.

\subsubsection{Diagnostics}
Various diagnostics may be issued, read as required from the message file, which by default is in\linebreak[30] \filename{\helpdirname/btrest.help}.

The most important ones are that it is not running and that the user is not permitted to invoke the command.


\subsection{\BtconnName}

\begin{expara}

\BtconnName{} hostname

\end{expara}

\PrBtconn{} instructs the \ProductName{} scheduler to attempt to raise a connection to the given host, which should not be currently active.


\subsection{\BtdisconnName}

\begin{expara}

\BtdisconnName{} hostname

\end{expara}

\PrBtdisconn{} instructs the \ProductName{} scheduler to close a connection to the given host, which should be currently active.


\subsection{\BtcichangeName}

\begin{expara}

\BtcichangeName{} [-options] name

\end{expara}

\PrBtcichange{} is a shell-level command to create, delete or change details of a command interpreter according to the
options specified. Only one command interpreter may be operated upon at a time.

The command interpreter in question is that given by the final argument name to the command.

The user must have \textit{special create} permission to operate this command - see \PrBtuser{} on page \pageref{btuser:permsreq}.

\subsubsection{Options}
The environment variable on which options are supplied is \filename{\BtcichangeVarname} and the environment variable to specify
the help file is \filename{BTRESTCONF}.

\setbkmkprefix{btcichange}
\explainopt

\cmdoption{A}{add}{}{add}

The command interpreter whose name and details are given with the other options is to be added.

\cmdoption{a}{args}{args}{args}

Set the ``predefined argument list'' to be that given by \genericargs{args}.

This replaces any existing predefined arguments.

Supply an empty string with {\textquotedbl}{\textquotedbl} to delete all arguments.

Almost invariably with shells, the \exampletext{{}-s} option should be supplied as a predefined argument. This will cause the
``real'' arguments supplied by the job, e.g. with the \exampletext{{}-a} option to \PrBtr{}, which follow the predefined arguments, to
be treated as strings and not the names of files.

\cmdoption{D}{delete}{}{delete}

The specified command interpreter is to be deleted. Note that the first entry on the list, which is initialised on installation to be
the Bourne shell \exampletext{sh}, cannot be deleted.

\warnings{N.B. This is not subject to extensive checking to ensure that no job currently uses the specified command interpreter, so please check
first.}

\cmdoption{e}{expand-args}{}{expandargs}

Expand \$-prefixed environment variables, \genericargs{\~{}user} and backquote constructs in job scripts before invoking the command interpreter, rather than relying upon the command interpreter to do it.

\cmdoption{i}{set-arg0-name}{}{arg0name}

Argument 0 of the job, when running, often displayed as the process name in ps(1) output, is the name of the command interpreter.

This is the default.

\cmdoption{L}{load-level}{number}{loadlev}

Set the load level to \genericargs{number} to be the default for new jobs created with this command interpreter.

The default for new command interpreters if this option is not given is the special create load level given in the user's profile as displayed by
\PrBtuser{}.

Remember that this load level must be less than or equal to a user's maximum load level per job for him/her to make use of this.

\cmdoption{N}{nice}{number}{nice}

Set the \progname{nice} value to number for jobs run under this command interpreter.
This is an ``absolute'' nice number. Most OSes have a standard nice of 20, so set this to 20 to get that,
to higher numbers to get lower priority and to lower numbers to get higher priority.

\cmdoption{n}{new-name}{name}{newname}

Supply a new name name for an existing command interpreter.

\warnings{N.B. Beware that existing jobs referring to the old name will not be checked for or changed.}

\cmdoption{p}{path}{pathname}{path}

Set the path \genericargs{pathname} to be the process invoked as the command interpreter.

Note that \PrBtcichange{} does not attempt to verify the accuracy of this path name.

Environment variables etc are not expanded here and the full path name (starting from \filename{/}) should be given.

\cmdoption{t}{set-arg0-title}{}{settitle}

Arrange that argument 0 for the command interpreter, when the job is started, is set to the job
title. On many systems this will make the output of ps display the job title in the output instead of the command interpreter name.

\cmdoption{U}{update}{}{update}

The specified command interpreter is to have details changed as specified.

This is the default in the absence of other options.

\cmdoption{u}{no-expand-args}{}{noexpandargs}

Turn off expansion of environment variables etc in the scripts passed to the command interpreter, allowing that to do it.

\freezeopts{\filename{\BtcichangeVarname}}{STOP}

\subsubsection{Examples}
To change the nice value, load level and to specify that the job title will become the process name for jobs running under the \filename{sh} command
interpreter:

\begin{expara}

\BtcichangeName{} -N 22 -L 500 -t sh

\end{expara}

To add a new command interpreter using the Korn shell with the \exampletext{{}-s} option:

\begin{expara}

\BtcichangeName{} -A -N 25 -L 1500 -p /bin/ksh -a {\textquotesingle}-s{\textquotesingle} ksh

\end{expara}

The quotes around \exampletext{{}-s} are not necessary in this case, only if spaces are included.

To change the name to \exampletext{korn}

\begin{expara}

\BtcichangeName{} -n korn ksh

\end{expara}


\subsection{\BtcilistName}

\begin{expara}

\BtcilistName{} [-options]

\end{expara}

\PrBtcilist{} causes a list of command interpreters, optionally for a remote host, to be output on standard output.

\subsubsection{Options}
The environment variable on which options are supplied is \filename{\BtcilistVarname} and the environment variable to specify
the help file is \filename{BTRESTCONF}.

\setbkmkprefix{btcilist}
\explainopt

\cmdoption{Q}{host}{host}{host}

Specifies the host name, defaulting to the host being run, for which the listing is required.

To cancel a previously-specified host name, use a single minus sign as an argument, or the local host name.

\freezeopts{\filename{\BtcilistVarname}}{}


\subsection{Bthols}

\begin{expara}

\BtholsName{} [-C] [-d] [-r] [-s] year [file]

\end{expara}

\PrBthols{} is a shell-level program to display or set the holidays file for the given year.

The holidays are displayed or interpreted in the following format (as for UK in 2004)

\begin{expara}

January: 1

April: 9 12

May; 3 31

August: 30

December: 27 28

\end{expara}

The year is given as 4 digits, thus \exampletext{2014}. Output when displaying, the
default, is to standard output.

If setting (with the \exampletext{{}-s} option) the input is from standard input or the specified file name. Holidays are added to
the existing list for the year unless the \exampletext{{}-C} option is also given.

Month names may be given in abbreviated or full format, case-insensitive, but are displayed in full. The full and abbreviated
names are extracted from the help file, by default \filename{\helpdirname/btrest.help}.

\subsubsection{Options}
The environment variable on which options are supplied is \filename{\BtholsVarname} and the environment variable to specify the
help file is \filename{BTRESTCONF}.

\setbkmkprefix{bthols}
\explainopt

\cmdoption{C}{clear}{}{clear}

Relevant only when the \exampletext{{}-s} option is specified, clear the existing holidays for the given year before
applying the new ones.

\cmdoption{d}{display}{}{display}

Display the existing holidays.

This is the default if no options are given.

\cmdoption{s}{set}{}{set}

Set holidays from standard input.

\cmdoption{r}{no-clear}{}{noclear}

Cancel any previously-set \exampletext{{}-C} option.

\freezeopts{\filename{\BtholsVarname}}{STOP}


\subsection{\HosteditName}

\begin{expara}

\HosteditName{} [-o file] [-s arg] [-I] [ file ]

\XhosteditName{} [-o file] [-s arg] [-I] [ file ]

\end{expara}

\PrHostedit{} is a simple curses-based program to edit host tables for \hostsfile, the host table for \ProductName{}.

\PrXhostedit{} is a GTK+ alternative which may be available to you.

It knows about local addresses, using web servers to get the local address, Windows clients,
DHCP, trusted hosts (although this is now deprecated), manual connections, probes and timeouts.

That said, you make not need to use it for a straightforward network connection to another host as
it is no longer compulsory to have details in the hosts file.

Input is taken from standard input unless a file name is given, and output is to standard output unless the \exampletext{{}-o}
option is given.

Alternatively use the \exampletext{{}-I} option to edit a file in place.

Normally this would be run as follows:

\begin{expara}

\HosteditName{} -I \hostsfilename

\end{expara}

You will usually have to stop and restart \ProductName{} after you have done this so that all parts of the system ``know''
about the new hosts, however this may not be necessary in all cases, you may only have to ``\exampletext{kill -1}'' the process id of the
\progname{xbnetserv} process.

\subsubsection{Options}

The options for \progname{xhostedit} are the same as for \PrHostedit{}, except that the former may take options interpreted by GTK+ as well.

\setbkmkprefix{hostedit}

\cmdoption{o}{}{file}{output}

Output to the named file rather than Standard Output

\cmdoption{s}{}{char}{sort}

Where \genericargs{char} is \exampletext{h} or \exampletext{i}.

Sort display by host name or by IP address.

\cmdoption{I}{}{}{inplace}

Edit the named file, which must always be given but need not exist, in place.

\subsubsection[Commands]{Commands}
The following command keys are used from within the screen displayed by \HosteditName{}. As with other \ProductName{} commands, any commands which operate
upon an existing item will do so with the item to which the cursor is moved.

\begin{center}
\begin{tabular}{l l}
\exampletext{k} or cursor up & Move cursor up.\\
\exampletext{j} or cursor down & Move cursor down.\\
\exampletext{N} or next page & Scroll down a screenful.\\
\exampletext{P} or previous page & Scroll up a screenful.\\
\exampletext{q} & Quit and write hosts file.\\
\exampletext{a} & Create a new hosts entry.\\
\exampletext{c} & Edit the selected hosts entry.\\
\exampletext{d} & Delete the selected hosts entry.\\
\exampletext{l} & Edit the local address.\\
\exampletext{L} & Set the local address from the selected host.\\
\exampletext{w} & Set the local address from a web server.\\
\exampletext{u} & Set the default user name for DHCP clients.\\
\end{tabular}
\end{center}


\section{Querying/managing batch jobs from the command line}
\subsection{\BtjchangeName}

\begin{expara}

\BtjchangeName{} [-options] job number ...

\end{expara}

\PrBtjchange{} is a program to modify details of a job or jobs from a shell script or another program. Jobs are specified by
using the job number, as displayed by \PrBtr{} with the \exampletext{{}-v} (verbose) option, or as in the output of the first column of the \PrBtjlist{} command with default format.

Remote jobs should be specified by prefixing the job numbers with the host name thus:

\begin{expara}

host:1234

\end{expara}

It is not necessary to specify any leading zeroes.

Several jobs may be specified at once to apply the same set of changes to all of them at once.

\subsubsection{Options}
As supplied, the options to \BtjchangeName{} are more or less identical to those for \PrBtr{}, except that existing jobs have
their parameters changed from whatever they are to the specified parameters, and there is no ``default'', in
that mentioning an option means that the user requires an existing parameter for the job or jobs changed. For details of the syntax and
much of the meaning of the options, in most cases this is the same as for \PrBtr{} at the corresponding option.

It is a mistake not to specify any options at all.

The environment variable on which options are supplied is \filename{\BtjchangeVarname} and the environment variable to specify
the help file is \filename{BTRESTCONF}.

\setbkmkprefix{btjchange}
\explainopt

\cmdoption{2}{grace-time}{time}{gracetime}

This option sets or changes the second stage time of handling over-running jobs to time, in seconds (the argument may be any number of seconds, or given
as \genericargs{mm:ss} for minutes and seconds).

For more information, see the further documentation of this on \pageref{btr:gracetime}.

\cmdoption{9}{catch-up}{~}{catchup}

This option changes the ``if not possible'' action of the job or jobs to catch up - one run of a series of missed runs is
done when it is possible without affecting future runs.

\cmdoption{A}{avoiding-days}{~}{avoiding}

This option sets the days to avoid when the job or jobs are to be repeated automatically. The days to avoid option supersedes whatever
was in the job or jobs, unless a leading comma is given.

Thus if the existing days to avoid in the job are \exampletext{Sat} and \exampletext{Sun},

\begin{expara}

\BtrName{} -A Wed ...

\end{expara}

will change the days to avoid to be Wednesday only, whereas

\begin{expara}

\BtrName{} -A ,Wed ...

\end{expara}

will change the days to avoid to be Saturday, Sunday and Wednesday.

A single \exampletext{{}-} argument cancels the days to avoid parameter altogether, thus

\begin{expara}

\BtrName{} -A {}- ...

\end{expara}

For more information about this option, in particular about altering the names for the
days, see the corresponding option for \PrBtr{} on page \pageref{btr:avoiding}.

\cmdoption{a}{argument}{string}{arg}

Provide an argument string to the command interpreter.

This will be added to the end of the arguments already in the job or jobs or to arguments already provided using the \exampletext{{}-a} option already.

If you want to clear the argument list and start again, use the \exampletext{{}-e} (see page \pageref{btjchange:cancarg}) option first.

\cmdoption{B}{assignment-not-critical}{~}{anotcrit}

This marks subsequently-specified assignments (with the \exampletext{{}-s} option) as ``not critical'', meaning that the assignment will be ignored if
it contains a reference to a variable on a remote host which is offline or inaccessible.

This must precede (not necessarily immediately) the \exampletext{{}-s} options to which it is to be applied.

Note that this option on its own doesn't change anything in the job or jobs.

\cmdoption{b}{assignment-critical}{~}{acrit}

This marks subsequently-specified assignments (with the \exampletext{{}-s} option) as
``critical'', meaning that the job or jobs will not start if the assignment contains a reference to a variable on
a remote host which is offline or inaccessible.

This must precede (not necessarily immediately) the \exampletext{{}-s} options to which it is to be applied.

Note that this option on its own doesn't change anything in the job or jobs.

\cmdoption{C}{cancelled}{~}{cancelled}

Set the job or jobs to the ``cancelled'' state.

\cmdoption{c}{condition}{condition}{condition}

Add a condition to be satisfied before the job or jobs may run.

The condition is added to the end of the list of conditions in the job or jobs and ones added by previous \exampletext{{}-c} options, up to a maximum of 10
conditions.

To delete any existing conditions and start from scratch, use the \exampletext{{}-y} option first.

The format of the condition argument is decribed fully on page \pageref{btr:condfmt}.

\cmdoption{D}{directory}{directory}{directory}

This option resets the working directory for the job or jobs.

Various constructs are recognised in the directory name, see page \pageref{btr:directory} for more details.

\cmdoption{d}{delete-at-end}{~}{deleteatend}

This option cancels any repeat option of the job or jobs so that they will be deleted at the end of the run rather than repeated or kept.

\cmdoption{E}{reset-environment}{~}{resetenv}

\warnings{Note that this option is different from the} \exampletext{{}-E} \warnings{option for} \PrBtr{}.

This option causes the environment for the job or jobs to be that of the environment of the \PrBtjchange{} command.

\cmdoption{e}{cancel-arguments}{~}{cancarg}

This option cancels any arguments previously set up in the job or jobs.

You probably want to use this if you are changing the arguments, otherwise any arguments you specify with the \exampletext{{}-a} option will go on the end of existing arguments.

\cmdoption{F}{export}{~}{export}

This marks the job or jobs to be visible throughout the network or ``exported''.

The job won't actually run on another host unless you go further and make the job ``remote runnable'' with the
\exampletext{{}-G} option, as described on page \pageref{btjchange:fullexport}.

\cmdoption{f}{flags-for-set}{letters}{assflags}

This option provides a set of ``flags'' for subsequent assignment operators specified by the \exampletext{{}-s} option, indicating when they
should apply.

The argument letters should be some or all of \exampletext{SNEACR} for respectively Start, Normal exit,
Error exit, Abort, Cancel and Reverse.

Note that this option on its own doesn't change anything in the job or jobs.

\cmdoption{G}{full-export}{~}{fullexport}

This option marks the job or jobs to be visible throughout the network and potentially available to run on any machine.

\cmdoption{g}{set-group}{~}{group}

This option sets the group owner of the job or jobs to be \genericargs{group}.

Note that the setting of the group is done as a separate operation from any other changes. Depending upon whether the pre-existing and new
modes and ownership permit the various operations, this may need to be done before, after or interleaved with other changes for it to
succeed.

\cmdoption{H}{hold-current}{~}{holdcurrent}

This option selects the variant of the ``if not possible'' action for the job or jobs to ``hold current''. The next run is done
when possible, but the usual time is not adjusted.

Note that unlike with the ``catch up'' option described on page \pageref{btjchange:catchup},
subsequent runs are not omitted, the job will repeatedly run until all missed runs are completed.

\cmdoption{h}{title}{title}{title}

This option supplies a title for the job or jobs, setting it to the supplied \genericargs{title}.

In the absence of this argument the title will be that of the last part of the file name, if any.

Note that this may be done whilst the job or jobs are running.

The title may be a string of any length containing any printable characters, but colon should be avoided to avoid confusion with queue names.

If the title contains spaces or characters interpreted by the shell, it should be surrounded by quotes.

\cmdoption{I}{input-output}{redirection-spec}{redir}

This option specifies a redirection for the job or jobs. Note that the new redirection will be appended to the list of redirections in each job or as specified by preceding \exampletext{{}-I} options.

To clear any existing ones and start the list of redirections from scratch, precede them with the \exampletext{{}-Z}
option.

When the job is executed the redirections are handled in order from first to last.

The format of redirection specifications are described fully on page \pageref{btr:redirfmt}.

\cmdoption{i}{interpreter}{name}{interpreter}

This option resets the command interpreter for the job or jobs to be that specified by the name, which should already be defined.

The load level is also set to that for the specified interpreter, so if a \exampletext{{}-l} argument is to be specified,
it should \emphasis{follow} the \exampletext{{}-i} option.

\cmdoption{J}{no-advance-time-error}{~}{noadv}

This specifies that if the job exits with an error, the next time to run is not advanced according to the repeat specification if applicable.

\cmdoption{j}{advance-time-error}{~}{adv}

This specifies that if the job exits with an error, the next time to run is still advanced if applicable according to the repeat
specification.

\cmdoption{K}{condition-not-critical}{~}{cnotcrit}

This option marks subsequently specified conditions set with the \exampletext{{}-c} option as ``not critical'',
i.e. a condition dependent on a variable on an offline or otherwise inaccessible remote host will be ignored in
deciding whether a job may start.

This must precede (not necessarily immediately) the \exampletext{{}-c} options to which it is to be applied.

Note that this option on its own doesn't change anything in the job or jobs.

\cmdoption{k}{condition-critical}{~}{ccrit}

This option marks subsequently specified conditions set with the \exampletext{{}-c} option as ``critical'', i.e. a condition dependent on a
variable on an offline or otherwise inaccessible remote host will cause the job to be held up.

This must precede (not necessarily immediately) the \exampletext{{}-c} options to which it is to be applied.

Note that this option on its own doesn't change anything in the job or jobs.

\cmdoption{L}{ulimit}{value}{ulimit}

This option sets the \filename{ulimit} (maximum file size) value for the job or jobs to the value given.

Set a value of zero (the default) to indicate an unlimited value.

We strongly recommend that this option not be used as it easily causes a lot of unexpected problems.

\cmdoption{l}{loadlev}{number}{loadlev}

This option sets the load level of the job or jobs to be number. The user must have ``special create permission'' for this to differ from that of the
command interpreter.

The load level is also reset by the \exampletext{{}-i} (set command interpreter) option, so this option must be used after that has been specified
if it is to have any effect.

\cmdoption{M}{mode}{modes}{modes}

This option sets the permission of the job or jobs to be as specified.

The format of the mode argument is described fully on page \pageref{btjchange:modefmt}.

\warnings{Note that this is different from that for the \PrBtr{} command as it contains syntax to add and subtract from existing modes.}

\cmdoption{m}{mail-message}{~}{mail}

This option sets the flag whereby completion messages are mailed to the owner of the job. (They may anyway if the jobs output to standard
output or standard error and these are not redirected).

\cmdoption{N}{normal}{~}{normal}

This option sets the job or jobs to normal ``ready to run'' state, as opposed to ``cancelled'' as set by the \exampletext{{}-C} option.

\cmdoption{n}{local-only}{~}{localonly}

This option marks the job or jobs to be local only to the machines which they are queued on, cancelling any \exampletext{{}-F} or \exampletext{{}-G} options. They will not be visible or runnable on remote hosts.

\cmdoption{o}{no-repeat}{~}{norep}

This option cancels any repeat option of the job or jobs, so that the they will be run and retained on the queue marked ``done'' at the end.

\cmdoption{P}{umask}{value}{umask}

This option sets the \progname{umask} value of the job or jobs to the octal value given. The value should be up to 3 octal digits as
per the shell.

\cmdoption{p}{priority}{number}{priority}

This option sets the priority of the job or jobs to be \genericargs{number}, which must be in the range given by the user{\textquotesingle}s minimum
and maximum priority.

\cmdoption{q}{job-queue}{queuename}{queue}

This option sets a job queue name as specified on the job or jobs. This may be any sequence of printable characters.

\cmdoption{R}{reschedule-all}{~}{delayall}

This option sets the ``not possible'' action of the job or jobs to reschedule all - the run is done when it is possible
and subsequent runs are rescheduled by the amount delayed.

\cmdoption{r}{repeat}{repeat\_spec}{repeat}

This option sets the repeat option of the job or jobs as specified.

The format of the repeat argument is the same as for \PrBtr{} and is described fully on page \pageref{btr:repeatfmt}.

\cmdoption{S}{skip-if-held}{~}{skip}

This option sets the ``not possible'' action of the job or jobs to skip - the run is skipped if it could not be done at the specified
time.

\cmdoption{s}{set}{assignment}{assign}

This option sets an assignment on the job or jobs to be performed at the start and/or finish of the job or jobs as selected by a
previously-specified \exampletext{{}-f} option (see page \pageref{btjchange:assflags}).

This option is cumulative, and will add to the list of assignments already in the job or as specified by previous \exampletext{{}-s} options.

To clear all existing assignments and start from scratch, precede the assignments with the \exampletext{{}-z} option.

The format of the assignment argument as for \PrBtr{}, and is described fully on page \pageref{btr:assfmt}.

\cmdoption{T}{time}{time}{time}

This option sets the next run time or time and date of the job or jobs as specified.

\cmdoption{t}{delete-time}{time}{deltime}

This option sets a delete time for the specified job or jobs as a time in hours, after which it will be automatically deleted if this time has
elapsed since it was queued or last ran.

Set to zero (the default) to retain the job or jobs indefinitely.

\cmdoption{U}{no-time}{~}{notime}

This option cancels any time setting on the job or jobs.

\cmdoption{u}{set-owner}{user}{owner}

This option sets the owner of the job or jobs to be \genericargs{user}.

Note that the setting of the user is done as a separate operation from any other changes. Depending upon whether the pre-existing and new
modes and ownership permit the various operations, this may need to be done before, after or interleaved with other changes to
succeed.

\cmdoption{W}{which-signal}{sig}{whichsig}

This option is resets which signal is used to kill the job or jobs after the maximum run time has been exceeded.

\cmdoption{w}{write-message}{~}{write}

This option is used to indicate that completion messages are written to the owner's terminal if available.

\cmdoption{X}{exit-code}{range}{exits}

This option sets the normal or error exit code ranges for the job or jobs. THe format of the \genericargs{range} argument is as given for the
\PrBtr{} command on page \pageref{btr:exits}.

\cmdoption{x}{no-message}{~}{nomess}

This option resets both flags as set by the \exampletext{{}-m} and \exampletext{{}-w} options.

\cmdoption{Y}{run-time}{time}{runtime}

This option sets a maximum elapsed run time for the specified job or jobs.

The argument time is in seconds, which may be written as \genericargs{mm:ss} or \genericargs{hh:mm:ss}.

For more details, including the interaction with the other arguments \exampletext{{}-W} and \exampletext{{}-2},see the documentation of this option for
\PrBtr{} given on page \pageref{btr:runtime}.

\cmdoption{y}{cancel-condition}{~}{canccond}

This option deletes any conditions set up in the job or jobs or by previous \exampletext{{}-c} options.

\cmdoption{Z}{cancel-io}{~}{cancio}

This option deletes any redirections set up in the job or jobs or by previous \exampletext{{}-I} options.

\cmdoption{z}{cancel-set}{~}{cancass}

This option deletes any assignments set up in the job or jobs or by previous \exampletext{{}-s} options.

\freezeopts{\filename{\BtjchangeVarname}}{Stop}

\subsubsection{Mode arguments}
\bookmark{modefmt}

The argument to the \exampletext{{}-M} option provides for a wide variety of operations. Note that this differs from the syntax for the corresponding
option of the \PrBtr{} command on page \pageref{btr:modefmt}.

Each permission is represented by a letter, as follows:

\begin{center}
\begin{tabular}{l l}
\exampletext{R} & read permission\\
\exampletext{W} & write permission\\
\exampletext{S} & reveal permission\\
\exampletext{M} & read mode\\
\exampletext{P} & set mode\\
\exampletext{U} & give away owner\\
\exampletext{V} & assume owner\\
\exampletext{G} & give away group\\
\exampletext{H} & assume group\\
\exampletext{D} & delete\\
\exampletext{K} & kill\\
\end{tabular}
\end{center}
Each section of the mode (user, group, others) is represented by the prefixes \exampletext{U:}, \exampletext{G:} and
\exampletext{O:} and separated by commas.

For example:

\begin{expara}

{}-M U:RWSMPDK,G:RWSDK,O:RS

\end{expara}

would set the permissions for the user, group and others as given. If the prefixes are omitted, as in

\begin{expara}

{}-M RWSDK

\end{expara}

then all of the job, group and other permissions are set to the same value.

An alternative format allows permissions to be added to the existing permissions, thus

\begin{expara}

{}-M U:+WD,G:+D

\end{expara}

will add the relevant permissions to whatever is currently set.

Similarly permissions may be cancelled individually by constructs of the form:

\begin{expara}

{}-M G:-W,O:-RS

\end{expara}

If the same operation is to be done with two or more of \exampletext{U}, \exampletext{G} or
\exampletext{O}, the letters may be run together, for example

\begin{expara}

{}-M GO:+W

\end{expara}

\subsubsection{Note on mode and owner changes}
Changing various parameters, the mode (permissions), the owner and the group are done as separate operations.

In some cases changing the mode may prevent the next operation from taking place. In other cases it may need to be done first.

Similar considerations apply to changes of the owner and the group.

\PrBtjchange{} does not attempt to work out the appropriate order to perform the operations, the user should execute
separate \PrBtjchange{} commands in sequence to achieve the desired effect.


\subsection{\BtjdelName}

\begin{expara}

\BtjdelName{} [ -options ] job number ...

\end{expara}

\PrBtjdel{} provides a means of deleting batch jobs from the shell or a program, optionally killing running jobs if
required.

Jobs are specified by using the job number, as displayed by \PrBtr{} with the \exampletext{{}-v}
(verbose) option, or as in the output of the first column of the \PrBtjlist{} command with default format.

Remote jobs should be specified by prefixing the job numbers with the host name thus:

\begin{expara}

host:1234

\end{expara}

It is not necessary to specify any leading zeroes.

Appropriate error messages are displayed if the user attempts to delete a job which is either running or if the user does not have the
necessary permissions.

\subsubsection{Options}
The environment variable on which options are supplied is \filename{\BtjdelVarname} and the environment variable to specify the
help file is \filename{BTRESTCONF}.

\setbkmkprefix{btjdel}
\explainopt

\cmdoption{C}{command-prefix}{name}{cprefix}

Specify the given name as the prefix for the command file, followed by the job number, to be used by the
\exampletext{{}-u} option rather than the default of \exampletext{C} (which in turn may be changed by editing the
message file).

\cmdoption{D}{directory}{name}{directory}

Save unqueued jobs to name rather than the current directory when \PrBtjdel{} is invoked.

\cmdoption{d}{delete}{}{delete}

Cancel any previous \exampletext{{}-k} option to be the default of deleting jobs.

\cmdoption{e}{do-not-unqueue}{}{nounqueue}

Cancel the effect of a previous \exampletext{{}-u} or \exampletext{{}-X} option.

\cmdoption{J}{job-prefix}{name}{jprefix}

Specify the given \genericargs{name} as the prefix for the job file, followed by the job number, to be used by the
\exampletext{{}-u} option rather than the default of \exampletext{J} (which in turn may be changed by editing the
message file).

\cmdoption{K}{signal-number}{signal}{signal}

Apply \genericargs{signal} given to kill running job. Default is 15 (\filename{SIGTERM}).

\cmdoption{k}{do-not-delete}{}{nodelete}

Kill jobs only where applicable, do not delete.

\cmdoption{N}{no-force}{}{noforce}

Do not kill or delete running jobs (default).

\cmdoption{S}{sleep-time}{seconds}{sleeptime}

Monitor process state for \genericargs{seconds} seconds after killing (default 10 seconds).

\cmdoption{u}{unqueue}{}{unqueue}

Unqueue job(s) to the current directory (or as specified by the \exampletext{{}-D} option). Do not delete if \exampletext{{}-k} given.

\cmdoption{X}{xml-unqueue}{}{xmlunqueue}

As with \exampletext{{}-u}, but unqueue to just a job file in XML format. Note that the suffix \batchjobsuffix{} will be appended if not specified.

\cmdoption{Y}{force}{}{force}

Kill and delete running jobs.

\freezeopts{\filename{\BtjdelVarname}}{Stop}

\subsubsection{Examples}
To delete jobs even if running:

\begin{expara}

\BtjdelName{} -y 1237 avon:9371

\end{expara}

Kill a job without deleting it with signal 2
(\filename{SIGINT}).

\begin{expara}

\BtjdelName{} -K 2 -k 9120

\end{expara}

Take a copy of the job in a work directory without deleting it.

\begin{expara}

\BtjdelName{} -u -k -D \~{}/work -C spec -J script 9123

\end{expara}


\subsection{\BtjlistName}

\begin{expara}

\BtjlistName{} [-options] [job numbers]

\end{expara}

\PrBtjlist{} is a program to display a summary of the jobs (or to be precise the jobs visible to the user) on the standard output.

Each line of the output corresponds to a single job, and by default the output is generally similar to the default format of the jobs screen of
the \PrBtq{} command. The first field on each line (unless varied as below) is the numeric job number of the job, prefixed
by a machine name and colon if the job is on a machine other than the one \BtjlistName{} is run on, job thus:

\begin{expara}

3493

macha:9239

machb:19387

\end{expara}

This is the required format of the job number which should be passed to \PrBtjdel{} and \PrBtjchange{}.

Various options allow the user to control the output in various ways as described below. The user can limit the output to specific jobs by
giving the job numbers as additional arguments.

\subsubsection{Options}
The environment variable on which options are supplied is \filename{\BtjlistVarname} and the environment variable to specify the
help file is \filename{BTRESTCONF}.

\setbkmkprefix{btjlist}
\explainopt

\cmdoption{B}{bypass-modes}{}{bypassmodes}

Disregard all modes etc and print full details. This is provided for dump/restore scripts. It is only available to users with
\textit{Write Admin File} permission, otherwise it is silently ignored.

This option is now deprecated as \PrXbCjlist{} is now provided for the purpose for which this option was implemented.

\cmdoption{D}{default-format}{}{deffmt}

Revert the output format to the default format.

\cmdoption{F}{format}{string}{format}

Changes the output format to conform to the pattern given by the format \genericargs{string}. This is further described below.

\cmdoption{g}{just-group}{group}{group}

Restrict the output to jobs owned by the specified \exampletext{group}.

To cancel this argument, give a single \exampletext{{}-} sign as a group name.

The group name may be a shell-style wild card as described below.

\cmdoption{H}{header}{}{header}

Generate a header for each column in the output.

\cmdoption{L}{local-only}{}{localonly}

Display only jobs local to the current host.

\cmdoption{l}{no-view-jobs}{}{noview}

Cancel the \exampletext{{}-V} option and view job parameters rather than job scripts.

\cmdoption{N}{no-header}{}{noheader}

Cancel the \exampletext{{}-H} option. Do not print a header.

\cmdoption{n}{no-sort}{}{nosort}

Cancel the \exampletext{{}-s} option. Do not sort the jobs into the order in which they will run.

\cmdoption{q}{job-queue}{name}{queue}

Restricts attention to jobs with the queue prefix name. The queue may be specified as a pattern with shell-like wild cards as
described below.

To cancel this argument, give a single \exampletext{{}-} sign as a queue name.

The queue prefix is deleted from the titles of jobs which are displayed.

\cmdoption{R}{include-all-remotes}{}{remotes}

Displays jobs local to the current host and exported jobs on remote machines.

\cmdoption{r}{include-exec-remotes}{}{execremotes}

Displays jobs local to the current host and jobs on remote machines which are remote-executable, i.e. which might possibly be
executed by the current machine.

\cmdoption{S}{short-times}{}{shorttimes}

Displays times and dates in abbreviated form, i.e. times within the next 24 hours as times, otherwise dates. This option is
ignored if the \exampletext{{}-F} option is specified.

\cmdoption{s}{sort}{}{sort}

Causes the output to be sorted so that the jobs whose next execution time is soonest comes at the top of the list.

\cmdoption{T}{full-times}{}{fulltimes}

Displays times and dates in full. This option is ignored if the \exampletext{{}-F} option is specified.

\cmdoption{u}{just-user}{user}{user}

Restrict the output to jobs owned by the specified \exampletext{user}.

To cancel this argument, give a single \exampletext{{}-} sign as a user name.

The user name may be a shell-style wild card as described below.

\cmdoption{V}{view-jobs}{}{view}

Do not display job details at all, output the scripts (input to the command interpreter) on standard output for each of the jobs specified, or all jobs
if none are specified.

\cmdoption{Z}{no-null-queues}{}{nonull}

In conjunction with the \exampletext{{}-q} parameter, do not include jobs with no queue prefix in the list.

\cmdoption{Z}{null-queues}{}{null}

In conjunction with the \exampletext{{}-q} parameter, include jobs with no queue prefix in the list.

\freezeopts{\filename{\BtjlistVarname}}{}

\subsubsection{Wild cards}
Wild cards in queue, user and group name arguments take a format similar to the shell.

\begin{center}
\begin{tabular}{l p{12.133cm}}
\exampletext{*} & matches anything\\
\exampletext{?} & matches a single character\\
\exampletext{[a-mp-ru]} & matches any one character in the range of characters given\\
\end{tabular}
\end{center}

Alternatives may be included, separated by commas. For example

\begin{expara}

{}-q {\textquotesingle}a*{\textquotesingle}

\end{expara}

displays jobs with queue prefixes starting with \exampletext{a}

\begin{expara}

{}-q {\textquotesingle}[p-t]*,*[!h-m]{\textquotesingle}

\end{expara}

displays jobs with queue prefixes starting with \exampletext{p} to \exampletext{t} or ending with anything other than
\exampletext{h} to \exampletext{m}.

\begin{expara}

{}-u jmc,tony

\end{expara}

displays jobs owned by \exampletext{jmc} or \exampletext{tony}

\begin{expara}

{}-g {\textquotesingle}s*{\textquotesingle}

\end{expara}

displays jobs owned by groups with names starting \exampletext{s}.

You should always put quotes around arguments containing the wildcard characters, to avoid misinterpretation by the shell.

\subsubsection{Format codes}

The format string consists of a string containing the following character sequences, which are replaced by the corresponding job
parameters. The string may contain various other printing characters or spaces as required.

Each column is padded on the right to the length of the longest entry. If a header is requested, the appropriate abbreviation is obtained from
the message file and inserted.

\begin{tabular}{l p{14cm}}
\exampletext{\%\%} & Insert a single \exampletext{\%} character.\\
\exampletext{\%A} & Insert the argument list for job separated by commas.\\
\exampletext{\%a} & Insert the ``days to avoid'' separated by commas.\\
\exampletext{\%b} & Display job start time or time job last started.\\
\exampletext{\%C} & Display conditions for job in full, showing operations and constants.\\
\exampletext{\%c} & Display conditions for job with variable names only.\\
\exampletext{\%D} & Working directory for job.\\
\exampletext{\%d} & Delete time for job (in hours).\\
\exampletext{\%E} & Environment variables for job. Note that this may make the output lines extremely long.\\
\exampletext{\%e} & \exampletext{Export} or \exampletext{Rem-runnable} for exported jobs.\\
\exampletext{\%f} & Last time job finished, or blank if it has not run yet.\\
\exampletext{\%G} & Group owner of job.\\
\exampletext{\%g} & Grace time for job (time after maximum run time to allow job to finish before final kill) in minutes and seconds.\\
\exampletext{\%H} & Title of job including queue name (unless queue name restricted with \exampletext{{}-q} option).\\
\exampletext{\%h} & Title of job excluding queue name.\\
\exampletext{\%I} & Command interpreter.\\
\exampletext{\%i} & Process identifier if job running, otherwise blank. This is the process identifier on whichever processor is running the job.\\
\exampletext{\%k} & Kill signal number at end of maximum run time.\\
\exampletext{\%L} & Load level\\
\exampletext{\%l} & Maximum run time for job, blank if not set.\\
\end{tabular}

\begin{tabular}{l p{14cm}}
\exampletext{\%M} & Mode as a string of letters with \exampletext{U:}, \exampletext{G:} or \exampletext{O:} prefixes as in
\exampletext{U:RWSMPUVGHDK,G:RSMG,O:SM}.\\
\exampletext{\%m} & Umask as 3 octal digits.\\
\exampletext{\%N} & Job number, prefixed by host name if remote.\\
\exampletext{\%O} & Originating host name, possibly different if submitted via \PrRbtr{} or the API.\\
\exampletext{\%o} & Original date or time job submitted.\\
\exampletext{\%P} & Job progress code, \exampletext{Run}, \exampletext{Done} etc.\\
\exampletext{\%p} & Priority\\
\exampletext{\%q} & Job queue name\\
\exampletext{\%R} & Redirections\\
\exampletext{\%r} & Repeat specification\\
\exampletext{\%S} & Assignments in full with operator and constant\\
\exampletext{\%s} & Assignments (variable names only)\\
\exampletext{\%T} & Date and time of next execution\\
\exampletext{\%t} & Abbreviated date or time if in next 24 hours\\
\exampletext{\%U} & User name of owner\\
\exampletext{\%u} & Ulimit (hexadecimal)\\
\exampletext{\%W} & Start time if running, end time if just finished, otherwise next time to run\\
\exampletext{\%X} & Exit code ranges\\
\exampletext{\%x} & Last exit code for job\\
\exampletext{\%Y} & If ``avoiding holidays'' is set, display holiday dates for the next year\\
\exampletext{\%y} & Last signal number for job\\
\end{tabular}

Note that the various strings such as \exampletext{export} etc are read from the message file also, so it is possible to modify them
as required by the user.

Only the job number, user, group, originating host and progress fields will be non-blank if the user may not read the relevant job. The mode
field will be blank if the user cannot read the modes.

The default format is

\begin{expara}

\%N \%U \%H \%I \%p \%L \%t \%c \%P

\end{expara}

with the (default) \exampletext{{}-S} option and

\begin{expara}

\%N \%U \%H \%I \%p \%L \%T \%c \%P

\end{expara}

with the \exampletext{{}-T} option.

\subsubsection{Examples}
The default output might look like this:

\begin{expara}

15367 jmc \ Go-to-optician \ memo 150 100 \ 10/08

25874 uucp dba:Admin \ \ \ \ \ \ sh \ \ 150 1000 11:48 \ \ \ \ \ Done

25890 uucp dba:Uuclean \ \ \ \ sh \ \ 150 1000 23:45

25884 uucp dba:Half-hourly sh \ \ 150 1000 10:26 Lock

26874 adm

\end{expara}

If the user does not have read permission on a job, then only limited information is displayed.

This might be limited to a different format with only jobs in queue dba as follows:

\begin{expara}

\$ \BtjlistName{} -q dba -Z -H -F {\textquotedbl}\%N \%H \%P{\textquotedbl}

Jobno Title \ \ \ \ \ \ Progress

25874 Admin \ \ \ \ \ \ Done

25890 Uuclean

25884 Half-hourly

\end{expara}


\subsection{\BtjstatName}

\begin{expara}

\BtjstatName{} [-options] jobnumber

\end{expara}

\PrBtjstat{} is provided to enable shell scripts to determine the status of a single job.

The jobs is specified by using the job number, as displayed by \PrBtr{} with the \exampletext{{}-v}
(verbose) option, or as in the output of the first column of the \PrBtjlist{} command with default format.

A remote job should be specified by prefixing the job number with the host name thus:

\begin{expara}

host:1234

\end{expara}

It is not necessary to include any leading zeroes.

By default, the job is checked to see if it is running, just starting or just finishing, but by means of the \exampletext{{}-s} option,
the user can specify which states to test for.

\PrBtjstat{} returns an exit code of 0 (true to shells) if the job is in the given state, 1 if it is not, and some
other exit code (and a diagnostic) if some other error occurs, e.g. the job does not exist.

\subsubsection{Options}
The environment variable on which options are supplied is \filename{\BtjstatVarname} and the environment variable to specify the
help file is \filename{BTRESTCONF}.

\setbkmkprefix{btjstat}
\explainopt

\cmdoption{d}{default-states}{}{defstates}

Cancel a \exampletext{{}-s} option and revert to checking whether the job is running, just starting or just finishing.

\cmdoption{s}{state}{statcodes}{state}

Specify \genericargs{statecodes} as the states to be tested for. \genericargs{Statecodes}
is a comma-separated list of states exactly as reported by \PrBtjlist{}.

The strings are read from the message file, and can be altered if required.

As distributed, they are

\begin{center}
\begin{tabular}{l l}
\textit{(empty)} & Ready to run \\
\exampletext{Done} & Normal exit \\
\exampletext{Err} & Error exit \\
\exampletext{Abrt} & Aborted \\
\exampletext{Canc} & Cancelled \\
\exampletext{Init} & Startup stage 1 (included in the default case) \\
\exampletext{Strt} & Startup stage 2 (included in the default case) \\
\exampletext{Run} & Running (included in the default case) \\
\exampletext{Fin} & Terminating (included in the default case) \\
\end{tabular}
\end{center}

\freezeopts{\filename{\BtjstatVarname}}{STOP}

\subsubsection{State names}

The state names are case insensitive. If one (typically the ``ready to run'' state) is a null string,
then this can be tested for by using a null string or two consecutive commas, thus:

\begin{expara}

\BtjstatName{} -s {\textquotesingle}{\textquotesingle} ...

\BtjstatName{} -s ,Done,Err ...

\BtjstatName{} -s Done,,Err ...

\end{expara}

\subsubsection{Example}
The following shell script displays a list of the titles of jobs ready to run or running

\pagebreak[11]
\begin{expara}

\BtjlistName{} -F {\textquotesingle}\%N \%H{\textquotesingle}{\textbar}while
read num title

do

\ \ \ \ if \BtjstatName{} -s {\textquotesingle}{\textquotesingle} \$num

\ \ \ \ then

\ \ \ \ \ \ \ \ echo \$title is ready to run

\ \ \ \ elif \BtjstatName{} \$num

\ \ \ \ then

\ \ \ \ \ \ \ \ echo \$title is running

\ \ \ \ fi

done

\end{expara}


\subsection{\BtjgoName, \BtjgoadvName, \BtjadvName}

\begin{expara}

\BtjgoName{} job number ...

\BtjgoadvName{} job number ...

\BtjadvName{} job number ...

\end{expara}

\PrBtjgo{} forces a job or jobs to run, ignoring the ``next run time''. Conditions and load level
constraints are however still enforced. The ``next run time'' will not be affected when the job completes. This
inserts an extra run of the job.

\PrBtjgoadv{} forces a job or jobs to run, ignoring the ``next run time''. Conditions and load
level constraints are however still enforced. The ``next run time'' is advanced to the next time. This brings
forward the next run, thereafter resuming the sequence.

\PrBtjadv{} advances the run time on each job specified to the next run time according to its repeat time without
running the job or looking at conditions.

These programs are all links to the \PrBtjdel{} binary. They all parse the same options as \PrBtjdel, but do not do anything with them.

Jobs are specified by using the job number, as displayed by \PrBtr{} with the \exampletext{{}-v}
(verbose) option, or as in the output of the first column of the \PrBtjlist{} command with default format.

Remote jobs should be specified by prefixing the job numbers with the host name thus:

\begin{expara}

host:1234

\end{expara}

It is not necessary to specify any leading zeroes.


\subsection{\BtdstName}

\begin{expara}

\BtdstName{} [ -R ] startdate enddate adjustment

\end{expara}

\PrBtdst{} adjusts all jobs between the specified start and end dates and times by adding the specified (possibly signed)
adjustment in seconds to it.

The dates and times may be specified in the forms

\begin{expara}

dd/mm

mm/dd

yy/mm/dd

\end{expara}

Which of the first two forms is chosen is taken from the existing time zone. For time zones greater or equal to 4 West from GMT, the
\genericargs{mm/dd} form is chosen, otherwise \genericargs{dd/mm}.

The dates may be followed by a comma and a time in the form \genericargs{hh:mm}, otherwise midnight is assumed.

When working out what to do, remember that Unix internal time is based upon Greenwich Mean Time (GMT), it is the display which changes, so
that the effect of moving the clocks forward is to make the times (held as GMT) appear later than they did before.

A negative adjustment is subtracted from the time, making jobs run sooner. This is therefore appropriate when the clocks go forward at the
start of the summer time. Likewise a positive adjustment should be used at the end of summer time.

The optional argument \exampletext{{}-R} tries to apply the option to all exported remote jobs, but this really is not recommended
as the local jobs on those hosts will be unaffected probably leaving the users on those machines confused.


\section{Querying/managing variables from the command line}
\subsection{\BtvlistName}

\begin{expara}

\BtvlistName{} [ -options ] [ variable names ]

\end{expara}

\PrBtvlist{} is a program to display \ProductName{} variables on the standard output. It can be used in both shell scripts and other programs.
Each line of the output corresponds to a single variable, and by default the output is generally similar to the default format of the
variables screen of the \PrBtq{} command. The first field on each line is the variable name prefixed by a machine name and colon thus:

\begin{expara}

macha:v1

machb:xyz

\end{expara}

if the variable is on a remote machine. This is the required format of the variable name which should be passed to \PrBtvar{} and other shell interface commands.

An example of the output of \PrBtvlist{} is as follows:

\begin{exparasmall}

CLOAD \ \ \ \ \ \ \ \ 0 \ \ \ \ \ \ \ \ \ \ \ \ \ \ \ \# Current value
of load level

Dell:CLOAD \ \ \ 0 \ \ \ \ \ \ \ \ Export \# Current value of load
level

arnie:CLOAD \ \ 1000 \ \ \ \ \ Export \# Current value of load level

LOADLEVEL \ \ \ \ 20000 \ \ \ \ \ \ \ \ \ \ \ \# Maximum value of load
level

LOGJOBS \ \ \ \ \ \ \ \ \ \ \ \ \ \ \ \ \ \ \ \ \ \ \ \# File to save
job record in

LOGVARS \ \ \ \ \ \ \ \ \ \ \ \ \ \ \ \ \ \ \ \ \ \ \ \# File to save
variable record in

MACHINE \ \ \ \ \ \ sisko \ \ \ \ \ \ \ \ \ \ \ \# Name of current host

Dell:Neterr \ \ 0 \ \ \ \ \ \ \ \ Export \# Exit code from polling

STARTLIM \ \ \ \ \ 5 \ \ \ \ \ \ \ \ \ \ \ \ \ \ \ \# Number of jobs to
start at once

STARTWAIT \ \ \ \ 30 \ \ \ \ \ \ \ \ \ \ \ \ \ \ \# Wait time in seconds
for job start

Dell:Two \ \ \ \ \ 2 \ \ \ \ \ \ \ \ Export \#

bar \ \ \ \ \ \ \ \ \ \ 1 \ \ \ \ \ \ \ \ \ \ \ \ \ \ \ \#

foo \ \ \ \ \ \ \ \ \ \ 123 \ \ \ \ \ \ Export \# Testing

\end{exparasmall}

If the user has `reveal' but not `read' permission on a variable, the name only is displayed.

Various options allow the user to control the output in various ways as described below. The user can limit the output to specific variables by
giving the variable names as arguments following the options.

\subsubsection{Options}
The environment variable on which options are supplied is \filename{\BtvlistVarname} and the environment variable to specify the
help file is \filename{BTRESTCONF}.

\setbkmkprefix{btvlist}
\explainopt

\cmdoption{B}{bypass-modes}{}{bypass}

Disregard all modes etc and print full details. This is provided for dump/restore scripts.

It is only available to users with \textit{Write Admin File} permission such as \batchuser{} or \filename{root}, otherwise it is silently ignored.
This option is now deprecated as \PrXbCvlist{} is now provided for the purpose for which this option was implemented.

\cmdoption{D}{default-format}{}{deffmt}

Revert to the default display format, cancelling the \exampletext{{}-F} option.

\cmdoption{F}{format}{string}{format}

Changes the output format to conform to the pattern given by the format \genericargs{string}. This is further described below.

\cmdoption{g}{just-group}{group}{group}

Restrict the output to variables owned by the specified \genericargs{group}.

To cancel this argument, give a single \exampletext{{}-} sign as a group name.

The group name may be a shell-style wild card as described below.

\cmdoption{H}{header}{}{header}

Generate a header for each column in the output.

\cmdoption{L}{local-only}{}{localonly}

Display only variables local to the current host.

\cmdoption{R}{include-remotes}{}{remotes}

Displays variables local to the current host and exported variables on remote machines.

\cmdoption{u}{just-user}{user}{user}

Restrict the output to variables owned by the specified \genericargs{user}.

To cancel this argument, give a single \exampletext{{}-} sign as a user name.

The user name may be a shell-style wild card as described below.

\freezeopts{\filename{\BtvlistVarname}}{}

\subsubsection{Format codes}
The format string consists of a string containing the following character sequences, which are replaced by the following variable
parameters. The string may contain various other printing characters or spaces as required.

Each column is padded on the right to the length of the longest entry.

If a header is requested, the appropriate abbreviation is obtained from the message file and inserted.

\begin{tabular}{l l}
\exampletext{\%\%} & Insert a single \exampletext{\%}.\\
\exampletext{\%C} & Comment field.\\
\exampletext{\%E} & Export if variable is exported\\
\exampletext{\%G} & Group owner of variable.\\
\exampletext{\%K} & Cluster if the variable is marked clustered\\
\exampletext{\%M} & Mode as a string of letters with \exampletext{U:},
\exampletext{G:} or \exampletext{O:} prefixes as in \exampletext{U:RWSMPUVGHD,G:RSMG,O:SM}.\\
\exampletext{\%N} & Name\\
\exampletext{\%U} & User name of owner.\\
\exampletext{\%V} & Value\\
\end{tabular}

Note that the various strings such as \exampletext{export} etc are read from the message file also, so it is possible to modify them
as required by the user.

Only the \exampletext{name}, \exampletext{user}, \exampletext{group}, \exampletext{export} and
\exampletext{cluster} fields will be non-blank if the user may not read the relevant variable. The mode field will be blank if the
user cannot read the modes.

The default format is

\begin{expara}

\%N \%V \%E \# \%C

\end{expara}


\subsection{\BtvarName}

\begin{expara}

\BtvarName{} [-options] variable name

\end{expara}

\PrBtvar{} is a shell level tool to display, create, delete, modify or test the values of \ProductName{}
variables. Testing may be ``atomic'', in the sense that if two or more users attempt to assign new values to the
same variable conditional on a test, only one will ``win''.

\subsubsection{Options}
The environment variable on which options are supplied is \filename{\BtvarVarname} and the
environment variable to specify the help file is \filename{BTRESTCONF}.

\setbkmkprefix{btvar}
\explainopt

\cmdoption{C}{create}{}{create}

Create the variable if it doesn't exist. An initial value should be supplied using the \exampletext{{}-s} option.

\cmdoption{c}{comment}{string}{comment}

Assign or update the given comment field of the variable to be \genericargs{string}.

\cmdoption{D}{delete}{}{delete}

Delete the variable.

\cmdoption{E}{set-export}{}{export}

Mark the variable as ``exported'', i.e. visible to other hosts.

\cmdoption{G}{set-group}{group}{group}

Change the group ownership of the variable to \genericargs{group}.

\cmdoption{K}{cluster}{}{cluster}

Set the ``clustered'' marker on the variable. When used in conditions or assignments, the local version is used. Note that this
option is set to be deprecated in future releases in favour of modifying the condition or assignment itself to select the local version.

\cmdoption{k}{no-cluster}{}{nocluster}

Reset the ``clustered'' marker on the variable.

\cmdoption{L}{set-local}{}{setlocal}

Mark the variable as local to the host only. This is the default for new variables, for existing variables it will turn off the
export flag if it is specified. To leave existing variables unaffected, invoke the \exampletext{{}-N} flag.

\cmdoption{M}{set-mode}{mode}{mode}

Set the mode (permissions) on the variable according to the \genericargs{mode} argument given.

\cmdoption{N}{reset-export}{}{resetexport}

Reset the \exampletext{{}-L} and \exampletext{{}-E} options. Don't change the state of the variable.

This means that existing variables are left unchanged and for new variables that they will be created local-only.

\cmdoption{o}{reset-cluster}{}{resetcluster}

Reset the \exampletext{{}-k} and \exampletext{{}-K} options.

For new variables this will restore to the default of not clustered. For existing variables this will mean that the cluster flag is left unchanged.

\cmdoption{S}{force-string}{}{forcestring}

Force all assigned values to string type even if they appear to be numeric.

\cmdoption{s}{set-value}{value}{setvalue}

Assign or initialise the variable with the given \genericargs{value}.

\cmdoption{U}{set-owner}{user}{owner}

Change the ownership of the variable to \genericargs{user}.

\cmdoption{u}{undefined-value}{value}{undefval}

In the test operations, if the variable does not exist, treat it as if it did exist and had the given \genericargs{value}.

\cmdoption{X}{cancel}{}{cancel}

Cancel options \exampletext{{}-S}, \exampletext{{}-C}, \exampletext{{}-D}, \exampletext{{}-s} and \exampletext{{}-u}.

\freezeopts{\filename{\BtvarVarname}}{}

\subsubsection{Conditions}
The six conditions \exampletext{+eq}, \exampletext{+ne}, \exampletext{+gt}, \exampletext{+ge}, \exampletext{+lt +le} followed by
a constant compare the variable value with the constant specified. The constant is assumed to be on the right of the comparison, for example:

\begin{expara}

\BtvarName{} +gt 4 myvar

\end{expara}

Returns an exit code of zero (``true'' to the shell) if \exampletext{myvar} is greater than 4, or 1 (``false'' to the shell) if it is less than
or equal to 4.

Some other exit code would be returned if \exampletext{myvar} did not exist or some error occurred.

This may be combined with other options, for example

\begin{expara}

\BtvarName{} -D +gt 100 myvar

\end{expara}

Would delete \exampletext{myvar} only if its value was greater than 100.

\begin{expara}

\BtvarName{} -s 1 +le 0 myvar

\end{expara}

Would assign 1 to \exampletext{myvar} only if its previous value was less than or equal to 0. Exit code 0 (shell
``true'') would be returned if the test succeeded and the other operation was completed successfully, exit code
1 (shell ``false'') would be returned if the test failed and nothing was done, or some other error if the variable
did not exist or the operation was not permitted.

The test is ``atomic'' in the sense that a diagnostic will occur, and no assignment made, if some other process
sets the value in between the test and the assignment (or other change).

The condition must follow all other options.

\exampletext{+eq}, \exampletext{+ne}, \exampletext{+lt} and \exampletext{+gt} may be represented as \exampletext{{}-e},
\exampletext{{}-n}, \exampletext{{}-l} and \exampletext{{}-g} but this is not particularly recommended, especially for the last two.

\subsubsection{Use of options}
With no options, then the current value of the variable is printed, for example:

\begin{expara}

\BtvarName{} abc

\end{expara}

prints out the value of variable \exampletext{abc}.

To assign a value, the \exampletext{{}-s} option should be used, thus

\begin{expara}

\BtvarName{} -s 29 abc

\end{expara}

assigns the numeric value 29 to \exampletext{abc}.

Remote variables are referred to as follows:

\begin{expara}

\BtvarName{} -s 32 host2:def

\end{expara}

assigns 32 to variable \exampletext{def} on \exampletext{host2}.

The conditional options should be the last to be specified.

The \exampletext{{}-u} option may be used to specify a value to substitute for a non-existent variable in a test rather than reporting
an error, for example:

\begin{expara}

\BtvarName{} -u 10 -gt 5 myvar

\end{expara}

will compare \exampletext{myvar} with 5 if it exists. If it does not exist, then it will compare the given value, in this case 10,
with 5, and in this case return ``true''. There should not be a diagnostic unless there is a completely different error.

\subsubsection{Note on mode and owner changes}
Changing various parameters, the mode (permissions), the owner and the group are done as separate operations.

In some cases changing the mode may prevent the next operation from taking place. In other cases it may need to be done first.

Similar considerations apply to changes of the owner and the group.

\PrBtvar{} does not attempt to work out the appropriate order to perform the operations, the user should execute
separate \PrBtvar{} commands in sequence to achieve the desired effect.


\section{Interactive job and variable administration}
\subsection{\BtqName}

\begin{expara}

\BtqName{} [ -options ]

\end{expara}

\PrBtq{} is an interactive program that allows the user to display in real-time state and details of
\ProductName{} jobs and variables on local or remote machines, refreshing the screen automatically as the queue
changes or variables are updated, and allowing the status of jobs and variables on the queue to be altered according to each
user's permissions and privileges.

Please see page \pageref{bkm:Btqdescr} for more details of the interactive commands.
This section focuses on the command-line options which may be used to control the initial display.

\subsubsection{Options}
The environment variable on which options are supplied is \filename{\BtqVarname} and the environment variable to specify the
help file is \filename{BTQCONF}.

Certain commands available on-screen enable many of these options to be changed and saved in configuration files.

\setbkmkprefix{btq}
\explainopt

\cmdoption{A}{no-confirm-delete}{}{noconfdel}

Suppress confirmation request for delete operations.

\cmdoption{a}{confirm-delete}{}{confdel}

Ask confirmation of delete operations (this is the default).

\cmdoption{B}{no-help-box}{}{nohelpbox}

Put help messages in inverse video rather than in a box (this is the default).

\cmdoption{b}{help-box}{}{helpbox}

Put help messages in a box rather than displaying inverse video.

\cmdoption{E}{no-error-box}{}{noerrorbox}

Put error messages in inverse video rather than in a box.

\cmdoption{e}{error-box}{}{errorbox}

Put error messages in a box rather than displaying inverse video.

\cmdoption{g}{just-group}{group}{group}

Restrict the output to jobs owned by the group specified.
Cancel this argument by giving a single \exampletext{{}-} sign
as an argument. The group name may be a pattern with shell-like wild cards.

\cmdoption{H}{keep-char-help}{}{keepchhelp}

When displaying a help screen, interpret the next key press as a command as well as clearing the help screen. This is the
default.

\cmdoption{h}{lose-char-help}{}{losechhelp}

Discard whatever key press is made to clear a help screen.

\cmdoption{j}{jobs-screen}{}{jobscreen}

Commence display in jobs screen. This is the default unless there are no jobs to be viewed.

\cmdoption{l}{local-only}{}{localonly}

Only display jobs or variables local to the machine.

\cmdoption{N}{follow-job}{}{followjob}

If the currently selected job or variable moves on the screen, try to follow it and do not try to retain relative screen positions.

\cmdoption{q}{job-queue}{name}{queue}

Restricts attention to jobs with the queue prefix \genericargs{name}.

Cancel this argument by giving a single \exampletext{{}-} sign as an argument. The queue name may be a pattern with shell-like wild
cards.

\cmdoption{r}{network-wide}{}{networkwide}

Display jobs on all connected hosts.

\cmdoption{s}{keep-cursor}{}{keepcursor}

If the currently-selected job or variable moves on the screen, try to preserve the
relative position of the cursor on the screen rather than following the
job or variable.

\cmdoption{u}{just-user}{user}{user}

Restrict the output to jobs owned by the user specified. Cancel this argument by giving a \exampletext{{}-} parameter.

The user name may be a pattern with shell-like wild cards.

\cmdoption{v}{vars-screen}{}{varsscreen}

Commence display in variables screen.

\cmdoption{Z}{no-null-queues}{}{nonull}

In conjunction with the \exampletext{{}-q} parameter, do not include jobs with no queue prefix in the list.

\cmdoption{z}{null-queues}{}{null}

In conjunction with the \exampletext{{}-q} parameter, include jobs with no queue prefix in the list.


\subsection{\XbtqName{} \XmbtqName}

\begin{expara}

\XbtqName{}

\XmbtqName{}

\end{expara}

\PrXbtq{} and \PrXmbtq{} are fully interactive GTK+ Motif alternatives to
the standard queue manager \PrBtq{}. As with
\PrBtq{} the format of the screen display, the help
messages and the command keystrokes can be easily altered to suit your
requirements.

Unlike \PrBtq{} there are no specific command line
options to \PrXmbtq{}. The facility to change or
specify resources settings for an X11 (and hence Motif) program on the
command line can be used.


\section{File Monitoring}
\subsection{\BtfilemonName}
\setbkmkprefix{btfilemon}
\bookmark{startdoc}
\begin{expara}

\BtfilemonName{} [-options]

\end{expara}

\PrBtfilemon{} executes a given program or script when specified files change in specified ways in a specified directory.

It is intentionally not integrated with the \ProductName{} core product, as there is no
automatic mechanism within Unix for signalling changes to files, and it is therefore necessary to ``poll'' or monitor
the files at a given interval. \PrBtfilemon{} is made as small as possible so that the ``polling'' does not have a large impact
on the system.

The rest of \ProductName{} is made to be ``event-driven'', as this has minimal impact on the system when the product is inactive.

The ``action'' of \PrBtfilemon{} may be to run a \ProductName{} job, set a variable, or perform some
completely unrelated task.

\PrBtfilemon{} may optionally be used to list or terminate running copies of itself.

\subsubsection{Options}
The GTK+ program \PrXfilemon{} and the X/Motif program \PrXmfilemon{} may be used to set
up the options to and run \PrBtfilemon{} rather than remembering them here.

The environment variable on which options are supplied is \filename{\BtfilemonVarname} and the environment variable to specify
the help file is \filename{FILEMONCONF}.

\explainopt

\cmdoption{A}{file-arrives}{}{arrives}

Perform the required action when a new file is detected in the directory.

\cmdoption{a}{any-file}{}{anyfile}

Perform the required action for any file name.

\cmdoption{C}{continue-running}{}{continue}

Continue \PrBtfilemon{} after a matching file and condition has been found, looking for further files.

\cmdoption{c}{script-command}{}{scriptcmd}

Specify command as a string to execute when one of the monitored events occurs. This is an alternative to \exampletext{{}-X}, which
runs a named shell script.

In the command the sequence \exampletext{\%f} is replaced by the name of the file whose activity has provoked the action, and
\exampletext{\%d} by the directory.

To use this option, be sure to enclose the whole shell command in quotes so that it is passed as one argument, thus:

\begin{expara}

xmessage -bg red {\textquotesingle}Found \%f{\textquotesingle}

\end{expara}

\cmdoption{D}{directory}{dir}{directory}

Specify the given \textit{dir} as the directory to monitor rather than the current directory.

\cmdoption{d}{daemon-process}{}{daemon}

Detach a further \PrBtfilemon{} as a daemon process to do the work, and return to the user.

\cmdoption{e}{include-existing}{}{inclexisting}

Include existing files in the scan, and report changes etc to those.

If the \exampletext{{}-A} option (watch for file arriving) is specified, this will have no effect unless an existing file is deleted
and is recreated.

\cmdoption{G}{file-stops-growing}{secs}{nogrow}

Activate command when a file has appeared, and has not grown further for at least \genericargs{secs}.

Distinguish this from the \exampletext{{}-M} option, which will check for any change, possibly in the middle of a file.

\cmdoption{I}{file-stops-changing}{secs}{nochange}

Activate command when a file has appeared, and has not been changed for at least \genericargs{secs}.

This is more inclusive than \exampletext{{}-M}, as it includes activities such as changing the ownership or mode of the file, or making hard links.

\cmdoption{i}{ignore-existing}{}{ignoreexisting}

Ignore existing files (default). However if an existing file is noted to have been deleted, and then re-created, the new version
will be treated as a new file.

\cmdoption{K}{kill-all}{}{killall}

Kill all \PrBtfilemon{} daemon processes belonging to the user, or all such processes if invoked by \filename{root}.

\cmdoption{k}{kill-processes}{dir}{killproc}

Kill any \PrBtfilemon{} daemon processes left running which are scanning the given directory. Processes must
belong to the invoking user, or \PrBtfilemon{} be invoked by \filename{root}.

\cmdoption{L}{follow-links}{}{follow}

Follow symbolic links to files (and subdirectories with th exampletext{{}-R} option).

\cmdoption{l}{list-processes}{}{list}

List running \PrBtfilemon{} processes and which directories they are accessing.

\cmdoption{M}{file-stops-writing}{secs}{nowrite}

Activate command when a file has appeared, and has not been written to, for at least \genericargs{secs}.

This is more inclusive than \exampletext{{}-G} as it includes writes other than to the end of the file.

It is less inclusive than \exampletext{{}-I} which also monitors for linking and permission-changing.

\cmdoption{m}{run-monitor}{}{runmonitor}

Run as a file monitor program (default) rather than listing processes as with \exampletext{{}-l} or killing monitor processes
as with \exampletext{{}-k} or \exampletext{{}-K}.

\cmdoption{n}{not-daemon}{}{notdaemon}

Do not detach \PrBtfilemon{} as a daemon process (default), wait and only return to the user when a file event has been detected.

\cmdoption{P}{poll-time}{secs}{polltime}

Poll directory every \genericargs{secs} seconds. This should be sufficiently small not to ``miss'' events for
a long time, but large enough to not load the system. The default if this is not specified is 20 seconds.

\cmdoption{p}{pattern-file}{pattern}{pattern}

Perform action on a file name matching \genericargs{pattern}.

The pattern may take the form of wild-card matching given by the shell, with \exampletext{*} \exampletext{?}
\exampletext{[a-z]} \exampletext{[!a-z]} having the same meanings as with the shell, and possible alternative patterns
separated by commas, for example:

\begin{expara}

{}-p {\textquotesingle}*.[chyl],*.obj{\textquotesingle}

\end{expara}

Remember to enclose the argument in quotes so that it is interpreted by \PrBtfilemon{} and not the shell.

\cmdoption{R}{recursive}{}{recursive}

Recursively follow subdirectories of the starting directory.

\cmdoption{r}{file-deleted}{}{deleted}

Perform action when a file matching the criteria has been deleted.

\cmdoption{S}{halt-when-found}{}{haltfound}

Halt \PrBtfilemon{} when the first matching file and condition has been found.

\cmdoption{s}{specific-file}{file}{specific}

Perform action only with a specific named file, not a pattern.

\cmdoption{u}{file-stops-use}{secs}{stopsuse}

Perform action when a file has appeared, and has not been read for at least \genericargs{secs}.

(Remember that many systems suppress or reduce the frequency with which this is updated).

\cmdoption{X}{script-file}{file}{script}

Specify the given script as a shell script to execute when one of the monitored events occurs.

This is an alternative to \exampletext{{}-c} where the whole command is spelled out.

The existence of the shell script is checked, and \PrBtfilemon{} will fail with an error message if it does not exist.

The shell script is passed the following arguments:

\begin{enumerate}
\item File name\item Directory path\item File size (or last file size if
file deleted).\item Date of file modification, change or access as
YYYY/MM/DD, but only for those type of changes.\item Time of file
modification, change or access as HH:MM:SS, but only for those type of
changes.\end{enumerate}

\freezeopts{\filename{\BtfilemonVarname}}{stop}

\subsubsection{File matching}

What to look for may be made to depend upon something happening to

\begin{tabular}{l l}
Any file & With the \exampletext{{}-a} option. Any file that meets the other criteria will trigger the event.\\
Specific file & With the \exampletext{{}-s} option, \PrBtfilemon{} will watch for the specific file named.\\
Pattern & With the \exampletext{{}-p} option, a file which matches the pattern and the other criteria will trigger the action.\\
\end{tabular}

\subsubsection{Criteria}
There are 6 criteria to watch for.

\begin{tabular}{l p{14cm}}
File arriving &
This is probably the most common case. If you want to wait for a file appearing and trigger an event, the -A option will look for this.\\
& \\
File removal &
This will watch for files being deleted, for example some applications use a ``lock file'' to denote
that they are being run, and you might wish to start something else when it has gone.
Remember that you might want to include existing files in the scan with \exampletext{{}-e} if the file in question existed when you
started \PrBtfilemon{}.\\
& \\
File stopped growing &
What this watches for is for a file being having been created,
or with the \exampletext{{}-e} option starting to ``grow'', and then apparently no longer grown
for the given time.
If files are arriving from FTP, for example, then when they are complete, they will cease to ``grow'' in
size.\\
& \\
File no longer written &
A file not used sequentially may be written to internally
rather than have additional data appended. This often occurs with
database files, where records are updated somewhere in the middle of
the file. If a series of database transactions is made and then
completed, the file will no longer be written to for some time, and
\BtfilemonName{} can be made to trigger an action after that time.
You will often want to include the \exampletext{{}-e} option if the file existed already on entry.\\
& \\
File no longer changed & This goes a stage further than ``no longer written'' as it includes any kind of change to the file,
such as permissions, owner, hard links or change of access and write times.\\
& \\
File no longer used & This monitors the access time of the file, updated whenever
the file is read, and proceeds when this has gone unchanged for the
specified time.
You will often want to include the \exampletext{{}-e} option with this if the file existed already on entry.\\
\end{tabular}

\subsubsection{Pre-existing files}
If the \exampletext{{}-i} (ignore existing) option is specified, which is the default, then no changes to existing files
which would otherwise match the criteria will be considered, except where an existing file is deleted and then recreated and
\PrBtfilemon{} ``notices'' this happen, in that the file is deleted before one ``poll'' of the directory and recreated
before another.

In other words, if the poll time is 20 seconds, then the deletion and recreation will have to be 20 seconds apart.

If the -e option to include existing files is specified, the \exampletext{{}-G} \exampletext{{}-u}
\exampletext{{}-M} \exampletext{{}-I} and \exampletext{{}-r} options will work as for new files but not
\exampletext{{}-A} as the file has already ``arrived''. However, if it is deleted, this
is ``noticed'' and then recreated, it will be treated as a ``new'' file.

\subsubsection{Recursive searches}
If recursive searches are specified using the \exampletext{{}-R} option, a separate
\PrBtfilemon{} process will be invoked for each subdirectory, for each further subdirectory within each of those
subdirectories, and for each new subdirectory created within one of those whilst each process is running, unless the
\exampletext{{}-r} option is being used to watch for file removal, whereupon only those subdirectories which existed to begin
with will be considered.

If the \exampletext{{}-S} option is specified to stop once a file has been found, each process will continue until a file is found
in its particular subdirectory.

\subsubsection{Examples}
Monitor the FTP directories for new files which have finished arriving, sending a message to the user:

\begin{expara}

\BtfilemonName{} -aRC -D /var/spool/ftp -G 30 -c {\textquotedbl}xmessage
{\textquotesingle}\%f in \%d{\textquotesingle}{\textquotedbl}

\end{expara}

Set a \ProductName{} variable to an appropriate
value when a file arrives in the current directory

\begin{expara}

\BtfilemonName{} -aAC -c {\textquotedbl}\BtvarName{} -s {\textquotesingle}\%f arrived{\textquotesingle} file\_var{\textquotedbl}

\end{expara}


\subsection{\XfilemonName{} and \XmfilemonName}

\begin{expara}

\XfilemonName{} \&

\XmfilemonName{} \&

\end{expara}

\PrXfilemon{} and \PrXmfilemon{} is a simple dialog interface for GTK+ and X/Motif to set up
parameters for \PrBtfilemon{}.


\section{User administration}
\subsection{\BtchargeName}

\begin{expara}

\BtchargeName{} [-options] [user] ...

\end{expara}

\PrBtcharge{} is now deprecated as charging is no longer supported.

If invoked, it prints a warning message and exits.


\subsection{\BtuchangeName}

\begin{expara}

\BtuchangeName{} [-options] [users]

\end{expara}

\PrBtuchange{} is a shell tool that may be used to update the user permissions file giving the user profiles of various
users and the operations which they may be permitted to perform within the \ProductName{} system. Alternatively the
{\textquotedbl}default permissions{\textquotedbl} may be updated. These are the permissions which are assigned by default to new
\ProductName{} users.

The invoking user must have \textit{write admin file} permission.

\subsubsection{Options}
The environment variable on which options are supplied is \filename{\BtuchangeVarname} and the environment variable to specify
the help file is \filename{BTRESTCONF}.

\setbkmkprefix{btuchange}
\explainopt

\cmdoption{A}{copy-defaults}{}{copydefall}

Copy the default profile to all users before setting other permissions on the named users (with the \exampletext{{}-u}
option) or after setting the defaults (with the \exampletext{{}-D} option).

The privileges of the invoking user are not changed by this operation.

\cmdoption{D}{set-defaults}{}{setdef}

Indicate that the other options are to apply to the default profile for new users.

\cmdoption{d}{default-priority}{num}{defpri}

Set the default job priority to \genericargs{num}, which must be between 1 and 255.

\cmdoption{J}{job-mode}{modes}{jobmode}

Set the default permissions on jobs according to the format of the \genericargs{modes} argument.

\cmdoption{l}{min-priority}{num}{minpri}

Set the minimum job priority to \genericargs{num}, which must be between 1 and 255.

\cmdoption{M}{max-load-level}{num}{maxload}

Set the maximum load level for any one job to \genericargs{num}, which must be between 1 and 32767.

\cmdoption{m}{max-priority}{num}{maxpri}

Set the maximum job priority to \genericargs{num}, which must be between 1 and 255.

\cmdoption{N}{no-rebuild}{}{norebuild}

Cancel the \exampletext{{}-R} option to rebuild the user permissions file.

This option is now deprecated.

\cmdoption{p}{privileges}{privileges}{privileges}

Set the privileges of the user(s) as specified by the argument.

\cmdoption{R}{rebuild-file}{}{rebuild}

Rebuild the user permissions file \filename{\spooldirname/btufile\IfXi{6}\IfGNU{1}} incorporating any changes in the password list.

This option is now deprecated and is ignored.

\cmdoption{S}{special-load-level}{num}{specll}

Set the special load level for the user(s) to num, which must be between 1 and 32767.

Note that this is irrelevant for users who do not have \textit{special create} privilege.

\cmdoption{s}{no-copy-defaults}{}{nocopy}

Cancel the effect of the \exampletext{{}-A}.

\cmdoption{T}{total-load-level}{num}{totll}

Set the total load level for the user(s) to \genericargs{num}, which must be between 1 and 32767.

\cmdoption{u}{set-users}{}{setu}

Indicate that the other options are to apply to the users specified on the rest of the command line, resetting any previous
\exampletext{{}-D} option.

\cmdoption{V}{var-mode}{mode}{varmode}

Set the default permissions on variables according to the format of the \genericargs{modes} argument.

\cmdoption{X}{dump-passwd}{}{dumppasswd}

Dump out the hash table of the password file to avoid re-reading the password file within the other programs.

This option is now deprecated and is ignored.

\cmdoption{Y}{default-passwd}{}{defpasswd}

Default handling of password hash file dump - rebuild if it is already present and \exampletext{{}-R} specified, otherwise not.

This option is now deprecated and is ignored.

\cmdoption{Z}{kill-dump-passwd}{}{killdump}

Delete any existing dumped password hash file.

This option is now deprecated and is ignored.

\freezeopts{\filename{\BtuchangeVarname}}{stop}

\subsubsection{Users or default}
In one operation \PrBtuchange{} either adjusts the default permissions, to be applied to new users, if
\exampletext{{}-D} is specified, or specified users, if nothing or \exampletext{{}-u} is specified. So first set the
required defaults:

\begin{expara}

\BtuchangeName{} -D -n 20 -p CR,SPC,ST,Cdft -A

\end{expara}

Then set named users (in fact changes to \filename{root} and \batchuser{} users are silently ignored).

\begin{expara}

\BtuchangeName{} -p ALL jmc root batch

\end{expara}

\subsubsection{Rebuilding the user control file}

The user control file is now held as ``default'' plus a list of ``differences'' so the options to do this are now deprecated and the
relevant options ignored.

\subsubsection{Dumping the password file}
This has now been deprecated and the options are ignored.

\subsubsection{Privileges}
The following may be specified as the argument to \exampletext{{}-p}, as one or more (comma-separated) of
argument may be one or more of the following codes, optionally preceded by a minus to turn off the corresponding privilege.

\begin{center}
\begin{tabular}{|l|l|}
\hline
\exampletext{RA} & read admin file\\\hline
\exampletext{WA} & write admin file\\\hline
\exampletext{CR} & create\\\hline
\exampletext{SPC} & special create\\\hline
\exampletext{ST} & stop scheduler\\\hline
\exampletext{Cdft} & change default\\\hline
\exampletext{UG} & or user and group modes\\\hline
\exampletext{UO} & or user and other modes\\\hline
\exampletext{GO} & or group and other modes.\\\hline
\end{tabular}
\end{center}

\exampletext{ALL} may be used to denote all of the permissions, and then perhaps to cancel some. For example:

\begin{expara}
{}-p CR,ST,Cdft

{}-p ALL,-WA

\end{expara}

A hexadecimal value is also accepted, but this is intended only for the benefit of the installation routines.

\subsubsection{Mode arguments}
The argument to the \exampletext{{}-J} and \exampletext{{}-V} options provides for a wide variety of operations.

Each permission is represented by a letter, as follows:

\begin{center}
\begin{tabular}{|l|l|}
\hline
\exampletext{R} & read permission\\\hline
\exampletext{W} & write permission\\\hline
\exampletext{S} & reveal permission\\\hline
\exampletext{M} & read mode\\\hline
\exampletext{P} & set mode\\\hline
\exampletext{U} & give away owner\\\hline
\exampletext{V} & assume owner\\\hline
\exampletext{G} & give away group\\\hline
\exampletext{H} & assume group\\\hline
\exampletext{D} & delete\\\hline
\exampletext{K} & kill (only valid for jobs)\\\hline
\end{tabular}
\end{center}

Each section of the mode (job, group, others) is represented by the prefixes \exampletext{U:}, \exampletext{G:} and
\exampletext{O:} and separated by commas.

For example:

\begin{expara}

{}-J U:RWSMPDK,G:RWSDK,O:RS

\end{expara}

would set the permissions for the user, group and others for jobs as given. If the prefixes are omitted, as in

\begin{expara}

{}-J RWSDK

\end{expara}

then all of the user, group and other permissions are set to the same value. Alternatively two of the \exampletext{J},
\exampletext{G} or \exampletext{O} may be run together as in

\begin{expara}

{}-J U:RWSKD,GO:RWS

\end{expara}

if ``group'' or ``other'' (in this case) are to have the same permissions.


\subsection{\BtulistName}

\begin{expara}

\BtulistName{} [-options] [user ...]

\end{expara}

\PrBtulist{} lists the permissions of users known to the \ProductName{} batch scheduler system. All users are listed
if no users are specified, otherwise the named users are listed. The report is similar to the main display of
\PrBtuser{}.

The invoking user must have \textit{read admin file} permission to use \PrBtulist{}.

\subsubsection{Options}
The environment variable on which options are supplied is \filename{\BtulistVarname} and the environment variable to specify the
help file is \filename{BTRESTCONF}.

\setbkmkprefix{btulist}
\explainopt

\cmdoption{D}{default-format}{}{deffmt}

Cancel the \exampletext{{}-F} option and revert to the default format.

\cmdoption{d}{default-line}{}{defline}

Display an initial line giving the default options (included by default).

\cmdoption{F}{format}{format}{format}

Format the output according to the format string given.

\cmdoption{g}{group-name-sort}{}{grpsort}

Sort the list of users by the group name in ascending order, then by users within that group as primary group.

\cmdoption{H}{header}{}{header}

Generate a header for each column of the output.

\cmdoption{N}{no-header}{}{nohdr}

Cancel the \exampletext{{}-H} option.

\cmdoption{n}{numeric-user-sort}{}{numusort}

Sort the list of users by the numeric user id (default).

\cmdoption{S}{no-user-lines}{}{nousers}

Suppress the user lines. It is an error to invoke this and the \exampletext{{}-s} option as well.

\cmdoption{s}{no-default-line}{}{nodef}

Suppress the initial line giving the default options. It is an error to invoke this and the \exampletext{{}-S} option as well.

\cmdoption{U}{user-lines}{}{ulines}

Display the user lines (default).

\cmdoption{u}{user-name-sort}{}{usort}

Sort the list of users by the user name.

\freezeopts{\filename{\BtulistVarname}}{}

\subsubsection{Format argument}

The format string consists of a string containing the following character sequences, which are replaced by various user permission
parameters. The string may contain various other printing characters or spaces as required.

Each column is padded on the right to the length of the longest entry.

If a header is requested, the appropriate abbreviation is obtained from
the message file and inserted.

\begin{center}
\begin{tabular}{|l|l|}
\hline
\exampletext{\%\%} & Insert a single \exampletext{\%} character.\\\hline
\exampletext{\%d} & Default priority\\\hline
\exampletext{\%g} & Group name\\\hline
\exampletext{\%j} & Job mode\\\hline
\exampletext{\%l} & Minimum priority\\\hline
\exampletext{\%m} & Maximum priority\\\hline
\exampletext{\%p} & Privileges\\\hline
\exampletext{\%s} & Special create load level\\\hline
\exampletext{\%t} & Total load level\\\hline
\exampletext{\%u} & User name.\\\hline
\exampletext{\%v} & Variable mode\\\hline
\exampletext{\%x} & Maximum load level\\\hline
\end{tabular}
\end{center}

The string \exampletext{DEFAULT} replaces the user name in the default values line, or the group name if the user name is not printed.
If the group name is not printed as well, then this will be omitted and will be indistinguishable from the rest of the output.

Note that the various strings are read from the message file, so it is possible to modify them as required by the user.

The default format is

\begin{expara}

\%u \%g \%d \%l \%m \%x \%t \%s \%p

\end{expara}

\subsubsection{Privileges format}
The following are output via the \%p format. Note that the actual strings are read from the message file, and are the same ones as are
used by \PrBtuchange.

\begin{center}
\begin{tabular}{|l|l|}
\hline
\exampletext{RA} & read admin file\\\hline
\exampletext{WA} & write admin file\\\hline
\exampletext{CR} & create\\\hline
\exampletext{SPC} & special create\\\hline
\exampletext{ST} & stop scheduler\\\hline
\exampletext{Cdft} & change default\\\hline
\exampletext{UG} & or user and group modes\\\hline
\exampletext{UO} & or user and other modes\\\hline
\exampletext{GO} & or group and other modes.\\\hline
\end{tabular}
\end{center}

\exampletext{ALL} is printed if all privileges are set.

\subsubsection{Modes}
Modes printed by the \%j and \%v options are as follows:

\begin{center}
\begin{tabular}{|l|l|}
\hline
\exampletext{R} & read permission\\\hline
\exampletext{W} & write permission\\\hline
\exampletext{S} & reveal permission\\\hline
\exampletext{M} & read mode\\\hline
\exampletext{P} & set mode\\\hline
\exampletext{U} & give away owner\\\hline
\exampletext{V} & assume owner\\\hline
\exampletext{G} & give away group\\\hline
\exampletext{H} & assume group\\\hline
\exampletext{D} & delete\\\hline
\exampletext{K} & kill (only valid for jobs)\\\hline
\end{tabular}
\end{center}

Each section of the mode (job, group, others) is represented by the prefixes \exampletext{U:}, \exampletext{G:} and
\exampletext{O:} and separated by commas.

For example:

\begin{expara}

U:RWSMPDK,G:RWSDK,O:RS

\end{expara}

This is exactly the same format as is expected by \PrBtuchange{} etc.


\subsection{\BtuserName}
\setbkmkprefix{btuser}

\begin{expara}

\BtuserName{} [ -options ]

\end{expara}

\PrBtuser{} provides 4 functions:

With no arguments it lists the permissions for the invoking user and exits.

\bookmark{permsreq}
With the \exampletext{{}-m} option it enables the invoking user to edit his own default job and variable permissions. The user must
have change default modes permission to do this.

With the \exampletext{{}-v} option it enables the invoking user to interactively review the list of permissions for all users. The user
must have read admin file permission to do this.

With the \exampletext{{}-i} option it enables the invoking user to interactively review and update the list of permissions for all
users. The user must have write admin file permission to do this.

Please see page \pageref{bkm:Btuserdescr} for more details of the interactive commands.
This section focuses on the command-line options which may be used to control the initial display.

\subsubsection{Options}
The environment variable on which options are supplied is \filename{\BtuserVarname} and the environment variable to specify the
help file is \filename{BTUSERCONF}.

Certain commands available on-screen enable many of these options to be changed and saved in configuration files.

\explainopt

\cmdoption{B}{no-help-box}{}{nohelpbox}

Put help messages in inverse video rather than in a box (this is the default).

\cmdoption{b}{help-box}{}{helpbox}

Put help messages in a box rather than displaying inverse video.

\cmdoption{d}{display-only}{}{disponly}

This is the default. A list of permissions is output to the standard output.

\cmdoption{E}{no-error-box}{}{noerrorbox}

Put error messages in inverse video rather than in a box.

\cmdoption{e}{error-box}{}{errorbox}

Put error messages in a box rather than displaying inverse video.

\cmdoption{g}{group-name-sort}{}{grpsort}

Sort the list of users by the group name in ascending order, then by users within that group as primary group.

This is only relevant to \exampletext{{}-v} or \exampletext{{}-i} options.

\cmdoption{H}{keep-char-help}{}{keepchhelp}

When displaying a help screen, interpret the next key press as a command as well as clearing the help screen. This is the
default.

\cmdoption{h}{lose-char-help}{}{losechhelp}

Discard whatever key press is made to clear a help screen.

\cmdoption{i}{update-users}{}{updusers}

View and update the list of users. This option requires \textit{write admin file} privilege.

\cmdoption{m}{set-default-modes}{}{setdef}

Interactively set the default modes for the invoking user. This option requires \textit{change default modes} privilege.

\cmdoption{n}{numeric-user-sort}{}{numusort}

Sort the list of users by the numeric user id (default).

\cmdoption{u}{user-name-sort}{}{usort}

Sort the list of users by the user name.

\cmdoption{v}{view-users}{}{viewusers}

View in read-only mode the list of users and permissions. This requires \textit{read admin file} privilege.


\subsection{\XbtuserName{} and \XmbtuserName}

\begin{expara}

\XbtuserName{} \&

\XmbtuserName{} \&

\end{expara}

\PrXbtuser{} and \PrXmbtuser{} are fully interactive Motif alternatives to the standard user control program \PrBtuser{}, but
only in the fully interactive mode, so it cannot be started by a user without \textit{Write admin file} privilege.

Unlike \PrBtuser{} there are no specific command line options to \PrXbtuser{} and \PrXmbtuser{} other than the ones
to interact with the X toolkit. The facility to change or specify resources settings for an X11 (and hence Motif) program on the
command line can be used.


\section{Web browser interface support}
The following commands are provided for the web browser interface support and are not documented further here.
\begin{center}
\begin{tabular}{|l|l|}
\hline
\progname{btjccgi} & Operations on jobs CGI program\\\hline
\progname{btjcgi} & List jobs CGI program\\\hline
\progname{btjcrcgi} & Create jobs CGI program\\\hline
\progname{btjdcgi} & Delete jobs CGI program\\\hline
\progname{btjvcgi} & View jobs CGI program\\\hline
\progname{btvccgi} & Operations on variables CGI program\\\hline
\progname{btvcgi} & List variables CGI program\\\hline
\progname{rbtjccgi} & Operations on jobs CGI program\\\hline
\progname{rbtjcgi} & List jobs CGI program\\\hline
\progname{rbtjcrcgi} & Create jobs CGI program\\\hline
\progname{rbtjdcgi} & Delete jobs CGI program\\\hline
\progname{rbtjvcgi} & View jobs CGI program\\\hline
\progname{rbtvccgi} & Operations on variables CGI program\\\hline
\progname{rbtvcgi} & List variables CGI program\\\hline
\end{tabular}
\end{center}
\section{System file management}
\subsection{\XbBtuconvName}

\begin{expara}

\XbBtuconvName{} [-D dir] [-v n] [-e n] [-s] [-f] user file outfile

\end{expara}

\PrXbBtuconv{} converts the \ProductName{} user file, which is usually
\filename{btufile}, possibly with a numeric suffix, held in the batch spool directory to an executable shell script file
\genericargs{outfile}, which if executed, would recreate the \ProductName{}
users' permissions with the same options and privileges.

\IfXi{\PrXbBtuconv{} understands the format of the saved user file for versions of \ProductName{} going back
to release 4, and when presented with a saved file, will attempt to work out from the format which release it relates to.}

In addition to options, two arguments are always supplied to \PrXbBtuconv{}.

\begin{tabular}{l p{14cm}}
User file & This is the file containing the attributes of the user permissions,
\filename{btufile\textit{n}} in the batch spool directory, by default \spooldir, or as relocated by re-specifying
\filename{SPOOLDIR}.\\
& \\
Output file & This file is created by \PrXbBtuconv{} to contain the executable shell script, containing
\PrBtuchange{} commands, which may be used to recreate the user file.
This file should be run before restarting the scheduler.\\
\end{tabular}

\subsubsection{Options}

Note that this program does not provide for saving options in \configurationfile{} or \homeconfigpath{} files.

\setbkmkprefix{xb-btuconv}

\cmdoption{D}{}{directory}{directory}

This option specifies the source directory for the users and user file. It can be specified as \exampletext{\$SPOOLDIR} or
\exampletext{\$\{SPOOLDIR-/var/spool/batch\}} etc and the environment and/or \linebreak[3]\masterconfig{} will be
interrogated to interpolate the value of the environment variable given.

If you use this, don't forget to put single quotes around it thus:

\begin{expara}

\XbBtuconvName{} -D {\textquotesingle}\$\{SPOOLDIR-\spooldirname\}{\textquotesingle} ...

\end{expara}

otherwise the shell will try to interpret the \exampletext{\$} construct and not \PrXbBtuconv.

\cmdoption{e}{}{n}{errorlim}

Tolerate \genericargs{n} errors of the kinds denoted by the other options before giving up trying to convert the file.

\cmdoption{f}{}{}{ignfmt}

Ignore format errors in the saved user file, up to the limit of errors given by the \exampletext{{}-e} option.

\cmdoption{s}{}{}{ignsize}

Ignore file size errors in the saved user file (up to the number of total errors given by the \exampletext{{}-e}
option.

\IfXi{\cmdoption{v}{}{number}{version}

Tell \PrXbBtuconv{} that the user file is for release number of \ProductName{}, where number
is 4 to 6. This may be necessary where the user file is corrupted and
\PrXbBtuconv{} cannot guess what is meant.}


\subsection{\XbCjlistName}

\begin{expara}

\XbCjlistName{} [-D dir] [-v n] [-e n] [-u] [-s] [-f] jobfile outfile workdir

\end{expara}

\PrXbCjlist{} converts \ProductName{} job files held in the batch spool
directory to an executable shell script which may be used to recreate
them. This may be useful for backup purposes or for one stage in upgrading from one release of \ProductName{} to
another.

The jobs are copied into the backup directory \exampletext{workdir}, and the generated shell script file
\exampletext{outfile} refers to files in that directory.

\IfXi{\PrXbCjlist{} understands the format of the saved job file for versions of \ProductName{} going back to
release 5, and when presented with a saved file, will attempt to work out from the format which release it relates to. If required, it will
skip apparent errors in the job file.}

In addition to options, three arguments are always supplied to \PrXbCjlist{}.

\begin{tabular}{l p{14cm}}
Job list file &
This is the file containing the attributes of the jobs,
\filename{btsched\_jfile} in the spool directory, by default \spooldir, or as relocated by re-specifying
\filename{SPOOLDIR}.\\
& \\
Output file & This file is created by \PrXbCjlist{} to
contain the executable shell script which may be used to create the
jobs.\\
& \\
Backup directory & This directory is used to hold the job data.\\
\end{tabular}

Note that saved jobs make use of the \exampletext{{}-U} option to \PrBtr{} to set the ownership correctly.

\subsubsection{Options}

Note that this program does not provide for saving options in \configurationfile{} or \homeconfigpath{} files.

\setbkmkprefix{xb-cjlist}

\cmdoption{D}{}{directory}{directory}

This option specifies the source directory for the jobs and job file
rather than the current directory. It can be specified as \exampletext{\$SPOOLDIR} or
\exampletext{\$\{SPOOLDIR-\spooldirname\}} etc and the environment and/or \linebreak[3]\masterconfig{} will be
interrogated to interpolate the value of the environment variable given.

If you use this, don't forget to put single quotes around it thus:

\begin{expara}

\XbCjlistName{} -D
{\textquotesingle}\$\{SPOOLDIR-\spooldirname\}{\textquotesingle}
...

\end{expara}

otherwise the shell will try to interpret the \exampletext{\$} construct and not \PrXbCjlist{}.

\cmdoption{e}{}{n}{errorlim}

Tolerate \genericargs{n} errors of the kinds denoted by the other options before giving up trying to convert the file.

\cmdoption{f}{}{}{ignfmt}

Ignore format errors in the saved jobs file, up to the limit of errors given by the \exampletext{{}-e} option.

\cmdoption{s}{}{}{ignsize}

Ignore file size errors in the saved jobs file (up to the number of total errors given by the \exampletext{{}-e}
option.

\cmdoption{u}{}{}{noucheck}

Do not check user names in the saved job file.

\cmdoption{I}{}{delimiter}{delimiter}

Rather than use a directory for saved jobs, put the job scripts inline as ``here'' documents in the shell script file, using
the specified delimiter and the job number to delimit the jobs.

\IfXi{\cmdoption{v}{}{number}{version}

Tell \PrXbCjlist{} that the jobs file is for release number of \ProductName{}, where number
is 4 to 6. This may be necessary where the user file is corrupted and \PrXbCjlist{} cannot guess what is meant.}


\subsection{\XbCjlistxName}

\begin{expara}

\XbCjlistxName{} [-v] [-u] [-D dir] [-j file] [-t n] [-o dir]

\end{expara}

\PrXbCjlistx{} converts \ProductName{} job files held in the batch spool
directory to a series of XML job files which may be resubmitted with \PrBts.
This may be useful for backup purposes or for one stage in upgrading from one release of \ProductName{} to
another.

\subsubsection{Options}

Note that this program does not provide for saving options in \configurationfile{} or \homeconfigpath{} files.

The default options are set up so that sensible results are achieved by invoking \PrXbCjlistx{} without any options. The jobs are copied to the
current directory.

\setbkmkprefix{xb-cjlistx}

\cmdoption{v}{}{}{verbose}

Give a blow-by-blow account of actions. Also set the verbose option on the saved jobs.

\cmdoption{u}{}{}{noucheck}

Ignore invalid user names in saved jobs rather than counting them as errors.

\cmdoption{D}{}{directory}{directory}

This option specifies the source directory for the jobs and job file
rather than the current directory. It can be specified as \exampletext{\$SPOOLDIR} or
\exampletext{\$\{SPOOLDIR-\spooldirname\}} etc and the environment and/or \linebreak[3]\masterconfig{} will be
interrogated to interpolate the value of the environment variable given.

If you use this, don't forget to put single quotes around it thus:

\begin{expara}

\XbCjlistxName{} -D
{\textquotesingle}\$\{SPOOLDIR-\spooldirname\}{\textquotesingle}
...

\end{expara}

otherwise the shell will try to interpret the \exampletext{\$} construct and not \PrXbCjlistx{}.

As a short cut, you can just use a word without any \exampletext{\$} construct, for example \exampletext{SPOOLDIR} to imply
\exampletext{\$SPOOLDIR}.

This is the default if no arguments are given, so \PrXbCjlistx{} will by default look for jobs
in \linebreak[3]\spooldir.

\cmdoption{j}{}{file}{jobfile}

Use the specified file as the job file, which by default is \filename{btsched\_jfile} in the spool directory, either \spooldir{} or as
specified in the \exampletext{{}-D} option.

If this option is not specified, \filename{btsched\_jfile} is assumed.

If a full path name is given to this option, and no \exampletext{{}-D} option is specified, then the directory part is taken as if it had
been supplied to \exampletext{{}-D} for example

\begin{expara}

\XbCjlistxName{} -j /var/batch/spool/jobfile

\end{expara}

will have the same effect as

\begin{expara}

\XbCjlistxName{} -D /var/batch/spool -j jobfile

\end{expara}

It is usually safe to omit both this and the \exampletext{{}-D} option to take a copy of the ``live'' files.

\cmdoption{o}{}{directory}{outdir}

Specify the directory to which job files will be copied. The current directory will be used if these are not specified.

\cmdoption{t}{}{n}{usetitle}

Create job file names using the first \textit{n} characters of the job titles of each job, ignoring non-alphanumeric characters and replacing spaces
with underscore, appending the XML job suffix of \batchjobsuffix.

If the result would clash with an existing file name, then \exampletext{\_nnn} sequences are inserted before the suffix until a unique name is
created.

If this option is not specified, then the file name is constructed out of the job number and the suffix \batchjobsuffix. This also happens if the
job does not have a title.


\subsection{\XbCvlistName}

\begin{expara}

\XbCvlistName{} [ -D dir ] [ -v n ] [ -e n ] [ -s ] [ -f ] var file outfile

\end{expara}

\PrXbCvlist{} converts \ProductName{} variables held in the batch spool
directory to an executable shell script which may be used to re-install
them. This may be useful for backup purposes or for one stage in
upgrade from one release of \ProductName{} to another.

\IfXi{\PrXbCvlist{} understands the format of the saved
variable file for versions of \ProductName{} going
back to release 4, and when presented with a saved file, will attempt
to work out from the format which release it relates to.}

In addition to options, two arguments are always supplied to \PrXbCvlist{}.

\begin{tabular}{l p{14cm}}
Variable list file &
This is the file containing the attributes of the variables, \filename{btsched\_vfile} in the batch spool directory, by
default \spooldir, or as relocated by re-specifying \filename{SPOOLDIR}.\\
& \\
Output file &
This file is created by \PrXbCvlist{} to contain the executable shell script, containing
\PrBtvar{} commands, which may be used to recreate the variables.\\
\end{tabular}

\subsubsection{Options}

Note that this program does not provide for saving options in \configurationfile{} or \homeconfigpath{} files.

\setbkmkprefix{xb-cvlist}

\cmdoption{D}{}{directory}{directory}

This option specifies the source directory for the variables
rather than the current directory. It can be specified as \exampletext{\$SPOOLDIR} or
\exampletext{\$\{SPOOLDIR-\spooldirname\}} etc and the environment and/or \linebreak[3]\masterconfig{} will be
interrogated to interpolate the value of the environment variable given.

If you use this, don't forget to put single quotes around it thus:

\begin{expara}

\XbCvlistName{} -D
{\textquotesingle}\$\{SPOOLDIR-/usr/spool/batch\}{\textquotesingle}
...

\end{expara}

otherwise the shell will try to interpret the \exampletext{\$} construct and not \PrXbCvlist{}.

\cmdoption{e}{}{n}{errorlim}

Tolerate \genericargs{n} errors of the kinds denoted by the other options before giving up trying to convert the file.

\cmdoption{f}{}{}{ignfmt}

Ignore format errors in the saved variables file, up to the limit of errors given by the \exampletext{{}-e} option.

\cmdoption{s}{}{}{ignsize}

Ignore file size errors in the saved variables file (up to the number of total errors given by the \exampletext{{}-e} option.

\IfXi{\cmdoption{v}{}{number}{version}

Tell \PrXbCvlist{} that the variables file is for release number of \ProductName{}, where number
is 4 to 6. This may be necessary where the user file is corrupted and \PrXbCvlist{} cannot guess what is meant.}


\subsection{\XbCiconvName}

\begin{expara}

\XbCiconvName{} [-D dir] [-v n] [-e n] [-u] [-s] [-f] cifile outfile

\end{expara}

\PrXbCiconv{} converts \ProductName{} command interpreters held in the
batch spool directory to an executable shell script which may be used
to re-install them. This may be useful for backup purposes or for one
stage in upgrade from one release of \ProductName{} to another.

\PrXbCiconv{} understands the format of the saved job file for versions of \ProductName{} going back to
release 4, and when presented with a saved file, will attempt to work out from the format which release it relates to.

In addition to options, two arguments are always supplied to \PrXbCiconv{}.

\begin{tabular}{l p{12cm}}
Command interpreter list file &
This is the file containing the attributes of the variables, \filename{cifile} in the batch spool directory, by default
\spooldir, or as relocated by re-specifying \filename{SPOOLDIR}.\\
& \\
Output file & This file is created by \PrXbCiconv{} to contain the executable shell script, containing
\PrBtcichange{} commands, which may be used to recreate the command interpreters.\\
\end{tabular}

\subsubsection{Options}

Note that this program does not provide for saving options in \configurationfile{} or \homeconfigpath{} files.

\setbkmkprefix{xb-ciconv}

\cmdoption{D}{}{directory}{directory}

This option specifies the source directory for the variables rather than the current directory. It can be specified as
\exampletext{\$SPOOLDIR} or
\exampletext{\$\{SPOOLDIR-\spooldirname\}} etc and the environment and/or \linebreak[3]\masterconfig{} will be
interrogated to interpolate the value of the environment variable given.

If you use this, don't forget to put single quotes around it thus:

\begin{expara}

\XbCiconvName{} -D
{\textquotesingle}\$\{SPOOLDIR-/usr/spool/batch\}{\textquotesingle}
...

\end{expara}

otherwise the shell will try to interpret the \exampletext{\$} construct and not \PrXbCiconv{}.

\cmdoption{e}{}{n}{errorlim}

Tolerate \genericargs{n} errors of the kinds denoted by the other options before giving up trying to convert the file.

\cmdoption{f}{}{}{ignfmt}

Ignore format errors in the saved variables file, up to the limit of errors given by the \exampletext{{}-e} option.

\cmdoption{s}{}{}{ignsize}

Ignore file size errors in the saved variables file (up to the number of total errors given by the \exampletext{{}-e}
option.

\IfXi{\cmdoption{v}{}{number}{version}

Tell \PrXbCiconv{} that the variables file is for release number of \ProductName{}, where number
is 4 to 6. This may be necessary where the user file is corrupted and \PrXbCiconv{} cannot guess what is meant.}


\subsection{\XbRipcName}

\begin{expara}

\XbRipcName{} [-d] [-r] [-F] [-A] [-D secs] [-P psarg] [-G] [-n] [-o file]

\ \ \ \ \ [-S dir] [-x] \ [-B n] [-N char]

\end{expara}

\PrXbRipc{} traces, and/or optionally monitors or deletes IPC facilities for \ProductName{}. Many of
the facilities are used for debugging, but it also serves as a quick method of deleting the IPC facilities, being easier to use than
\progname{ipcs} and \progname{ipcrm}, in the event that the scheduler has crashed or been killed without deleting
the IPC facilities.

To use this facility, just run \PrXbRipc{} thus:

\begin{expara}

\XbRipcName{} -d {\textgreater}/dev/null

\end{expara}

The diagnostic output may be useful as it reports any inconsistencies.

The monitoring option can be used to diagnose processes, possibly not \ProductName{} ones, which are interfering with
\ProductName{} shared memory segments, in cases where a third-party application is suspected of damaging the shared
memory.

\PrXbRipc{} also checks for errors in memory-mapped files where the version of \ProductName{} is using those rather than shared
memory.

\subsubsection{Options}

Note that this program does not provide for saving options in \configurationfile{} or \homeconfigpath{} files.

\setbkmkprefix{xb-ripc}

\cmdoption{A}{}{}{jvdets}

Display details of jobs and variables. This often generates a lot of output and is not really necessary.

\cmdoption{D}{}{secs}{pollshm}

Monitor which process has last attached to the job shared memory segment and report apparent corruption, polling every \genericargs{secs} seconds.

\cmdoption{d}{}{}{delete}

Delete the IPC facilities after printing out contents. This saves messing with arguments to \progname{ipcrm(1)}.

\cmdoption{f}{}{}{dispfree}

Display the free chains for jobs and variables. This generates a lot of output and isn't usually necessary.

\cmdoption{n}{}{}{nodispok}

Suppress display from \exampletext{{}-D} option if everything is OK.

\cmdoption{o}{}{outfile}{outfile}

Output to \textit{outfile} rather than standard output. Set it to \filename{/dev/null} if you don't want to see any
output.

\cmdoption{P}{}{psarg}{psarg}

Specify argument to \progname{ps(1)} to invoke if corruption detected when monitoring with \exampletext{{}-D}
option. The output is passed through \progname{fgrep(1)} to find the line (if any) with the process id of the process which last
attached to the shared memory.

\cmdoption{G}{}{}{argfull}

Used in conjunction with the \exampletext{{}-P} option, the output from \progname{ps(1)} is displayed in
full, without passing it through \progname{fgrep(1)}.

\cmdoption{r}{}{}{readq}

Read and display the entries on the message queue. This is normally suppressed because they can't be
``peeked at'' or ``unread''.

\cmdoption{S}{}{dir}{dir}

This is only relevant for versions of \ProductName{} which use memory-mapped files rather
than shared memory. It specifies the location of the spool directory.
If this is not specified, then the master configuration file \linebreak[3]\masterconfig is consulted to find the spool
directory location, or failing that, the directory \spooldir{} is used.

\cmdoption{x}{}{}{hexdump}

Dump the contents of shared memory or memory-mapped files in hexadecimal and ASCII characters.

\cmdoption{B}{}{n}{dumpwidth}

Where \genericargs{n} may be 1 to 8, specify the width of the hexadecimal dump output as a number of 32-bit words.

\cmdoption{N}{}{char}{replchar}

Replace the character in the ASCII part of the hexadecimal dump to represent non-ASCII characters with the specified character
(the first character of the argument). The default is \exampletext{.} To specify a space, you may need to use quotes
thus: \exampletext{{}-N {\textquotesingle} {\textquotesingle}}

\subsubsection{Example}
To delete all IPC facilities after \ProductName{} has crashed.

\begin{expara}

\XbRipcName{} -d -o /dev/null

\end{expara}

To monitor the job shared memory segment for errors, printing out the \progname{ps(1)} output (where the full listing is obtained
with \exampletext{{}-ef}) search for the process id of the last process to attach to the segment. Print out the contents of the
segment including in hexadecimal after corruption is detected.

\begin{expara}

\XbRipcName{} -D 30 -P -ef -o joblog -A -x

\end{expara}


\subsection{\XipasswdName}

\begin{expara}

\XipasswdName{} [-u user] [-p password] [-f] [-d] [-F file]

\end{expara}

\PrXipasswd{} sets a password for the current user or a specified user if \exampletext{{}-u} is given. This is
separate and distinct from the user's login password.
This password is used by the web interfaces, the Windows interfaces and
the APIs\IfXi{ for both \ProductName{} and \OtherProductName}.

The reason for doing this is because it is considered insecure to possibly repeatedly try login passwords from user programs.

If any users have a password set in this way, then all users will have to to have a password in the file to use any of the interfaces
requiring a password.

Unlike the Unix \progname{passwd(1)} routine, the old password is not prompted for and there is no confirmation.

\subsubsection{Options}

Note that this program does not provide for saving options in \configurationfile{} or \homeconfigpath{} files.

\setbkmkprefix{xipasswd}

\cmdoption{u}{}{user}{user}

Set password for given user. This may only be for other than the current user if \PrXipasswd{} is invoked by
\filename{root}.

\cmdoption{p}{}{passwd}{passwd}

Specify the password to be set other than prompting for it.

\cmdoption{f}{}{}{insist}

It is normally considered an error to include a password for \filename{root} for the same reasons that the password file is
separate. However this option may be set to insist upon it.

\cmdoption{d}{}{}{deluser}

Delete the user's password from the file.

\cmdoption{F}{}{file}{file}

Use \genericargs{file} to hold the password. The default if no file is given
is \filename{\IfXi{/usr/spool/progs/xipwfile}\IfGNU{/usr/local/share/gbpwfile}}. Any number of
\exampletext{{}-F} options may be given to set up several password files at once.


\IfXi{\section{Licence manipulation and installation programs}
\input{ucmds/xb-checklic.tex}
\input{ucmds/xb-emerglic.tex}
\input{ucmds/xb-triallic.tex}
\input{ucmds/xb-vwrite.tex}
\input{ucmds/whatvn.tex}
\input{ucmds/zinstall.tex}}

