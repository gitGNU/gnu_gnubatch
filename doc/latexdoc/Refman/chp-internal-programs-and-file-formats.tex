\chapter{Internal Programs and file formats}
\label{chp:internal-programs-and-file-formats}
The following lists the internal programs and file formats used by \ProductName{}. With a few exceptions these programs should not be invoked or
accessed by a user including an administrator.

The internal programs are all held in the same directory, \progsdir{} by default. If the default directory is not used it will be pointed to by the
\filename{SPROGDIR} environment variable set up in the \masterconfig{} file.

\section{Core Programs}
\subsection{btsched}

\begin{expara}

\progsdirname/btsched [-options]

\end{expara}

\progname{btsched} is the scheduler process for the \ProductName{} batch processing system.

It is normally invoked by the system startup routines, or otherwise by \PrBtstart{}.

It may take certain options from the command line, but these are normally passed to it by \PrBtstart{} and are not
documented here as they are part of the internal interfaces of \ProductName{} and are subject to change.

Information, either in respect of other machines to connect to, or pre-existing jobs and variables on the current machine, are read
respectively from the files \hostsfile{} and the directory \spooldir{} (unless changed via the \textit{master configuration file} \masterconfig).

A ``slave'' \progname{btsched} process is spawned to control running jobs, and if a networked version of \ProductName{} is being run,
then an additional ``slave'' btsched process is spawned to monitor and process incoming network messages.

Incoming remotely-submitted jobs and API interfaces are handled via a separate process (also invoked by \PrBtstart{}), \progname{xbnetserv}.

\subsubsection{Files used}
\begin{center}
\begin{tabular}{l l}
\hostsfile &
Host names and descriptions\\
\masterconfig & Master configuration file\\
\filename{\helpdirname/btint-config} & Message file\\
\spooldir & Spool directory\\
\filename{\spooldirname/btsched\_jfile} & Job file\\
\filename{\spooldirname/btsched\_vfile} & Variables file\\
\filename{\spooldirname/btsched\_reps} & Error log file\\
\filename{\spooldirname/btufile} & User data\\
\filename{\spooldirname/btmm\_jobs} & Job memory-mapped file\\
\filename{\spooldirname/btmm\_vars} & Variables memory-mapped file\\
\filename{\spooldirname/btmm\_xfer} & Communication buffer memory-mapped file\\
\end{tabular}
\end{center}
\subsubsection[IPC Facilities used]{IPC Facilities used}
An IPC message queue, with key \exampletext{0x5869b000} and owned by user \batchuser{} is created by
\progname{btsched} and used to receive messages from user processes and pass instructions to and internal messages from the slave
\progname{btsched} processes to the master.

Two shared memory segments are created to hold details of jobs and variables. As the shared memory facility provides no facilities for
growth, then additional shared memory segments may be created if the job and variable lists expand sufficiently and the original ones
deallocated.

A further shared memory segment, with key \exampletext{0x5869b100} is created to hold details of pending
jobs before transfer to the main shared memory segment.

The keys given to the shared memory segments start at \exampletext{0x5869b002} and ascend upwards to
\exampletext{0x5869b064} before wrapping around.

Some versions of \progname{btsched} may use memory-mapped files rather than shared memory.
The files are held in the spool directory, by default \spooldir, and have the names \filename{btmm\_jobs},
\filename{btmm\_vars} and \filename{btmm\_xfer}.

A set of at least 10 semaphores, with the key \exampletext{0x5869b001} is created to interlock access to the
shared memory segments, and a further set of 3 semaphores with the key \exampletext{0x5869b003} is created to interlock network
processes.

The presence or absence of these IPC facilities is used by \progname{btsched} and other programs to determine whether
a previous copy of itself is running. If \progname{btsched} is abnormally terminated, it may be necessary to delete these IPC
facilities before \progname{btsched} can be restarted.

The utility \PrXbRipc{} may be used to delete the IPC facilities quickly.

\subsubsection{Internet ports used}
\progname{btsched} accepts and sends interconnections from other machines on TCP port, passes the contents of batch jobs on a
further TCP port, and undertakes ``probes'' on a UDP port.

The port numbers are set up in the \filename{/etc/services} file when \ProductName{} is first installed.

\subsection{xbnetserv}

\begin{expara}

\progsdirname/xbnetserv

\end{expara}

\progname{xbnetserv} is the remote server process for the \ProductName{} batch scheduler system.

It serves 3 purposes

\begin{enumerate}
\item It accepts jobs from other hosts submitted by \PrRbtr{} and queues them on the same server.
\item It accepts jobs and administration requests from DOS and Windows machines.
\item It supports API operations.
\end{enumerate}
It is normally invoked by the system startup routines, or otherwise by \PrBtstart{}.

It takes no arguments from the command line (and ignores any which are supplied). Information, in respect of other machines to connect to is
read from the file \hostsfile.

\subsubsection{Internet ports}
\progname{xbnetserv} uses 2 ports, one to accept incoming jobs on TCP from \PrRbtr{} and one to accept jobs on UDP from MS Windows clients.

\IfXi{Please note that the ports are called \filename{xbnetsrv} (with no second``\exampletext{e}''). This is because some historic
Unix systems limited hostnames to 8 characters.}

There are also TCP and UDP ports to accept API requests.

The port numbers are set up in the \filename{/etc/services} file when \ProductName{} is first installed.

\subsubsection{Diagnostics}
\progname{xbnetserv} runs as a ``daemon process'' and diagnostics, apart from those detected when
it is first started, are not usually written to any terminal but to the file \filename{\spooldirname/btsched\_reps}.

In the event of any problems this file should be examined.

\section{btexec}

\begin{expara}

\progsdirname/btexec options

\end{expara}

\progname{Btexec} runs commands for macros under the identity of the invoking user.
This is required because \PrBtq{} and \PrXmbtq{} are set-user programs (to other than \filename{root}) and there is an inherent security breach in
many versions of Unix in that such programs cannot divest themselves of traces of the set-user user id.

This program is only intended for internal use and is not further documented.

\section{Message Handlers}
\subsection{btmdisp}

\begin{expara}

\progsdirname/btmdisp options

\end{expara}

\progname{btmdisp} generates messages as required by \progname{btsched} in response to the mail or write
completion options of batch jobs.

By default, it uses the system basic mailer to dispose of mail options, \progname{btwrite} to send messages
to users' terminals and \progname{dosbtwrite} to send messages to Windows PCs.

The messages are generated by \progname{btmdisp} from the system message file, which by default is
\filename{\helpdirname/btint-config}.

The program to be used in each case may be overridden by assignments to the environment variables \filename{MAILER},
\filename{WRITER} and \filename{DOSWRITER}, most conveniently in the master configuration file
\masterconfig. The program (or shell script) to be run in each case should take data on standard input and the
relevant user name as the first argument, and will run under the identity \batchuser{}.

These variables may also be set on a per-user basis by assignment in a \homeconfigpath{} file located off a user's
home directory. The user may also specify an alternative message file by assignment to the variable \filename{SYSMESG}. These
programs or scripts will be run under the identity of the user, typically the owner of the job to be run.

The interface (options etc) are internal to \ProductName{} and are not documented here.

\IfXi{\subsubsection{Notes}
\progname{btmdisp} is identical to
\OtherProductName's \progname{spmdisp},
apart from using different message and configuration files.}
\subsection{btwrite}

\begin{expara}

\progsdirname/btwrite user [ ... ]

\end{expara}

\progname{btwrite} sends messages to users' terminals in response to the \exampletext{{}-w} option of
\PrBtr{} and equivalent. It is used in preference to the \progname{write}(1) command as this picks just one (and
usually the wrong one!) of the terminals at which the user may be logged in, and does not display a suitable name for the originator of
messages.

\progname{btwrite} takes a list of one or more users as arguments. It sends the text on standard input to each
user's terminal. The message is mailed to users who cannot be reached. This facility is available for use in your own
software if you wish.

\IfXi{\subsubsection{Notes}
\progname{btwrite} is identical to \OtherProductName's spwrite(8), apart from using a different message file.}

\subsection{dosbtwrite}

\begin{expara}

\progsdirname/dosbtwrite options

\end{expara}

\progname{dosbtwrite} sends messages to Windows PCs similar to \progname{btwrite} does for user's terminals in response to the equivalents of the
\exampletext{{}-w} options of \PrBtr{} and equivalents. This is only done for jobs which originated on Windows PCs.

The Windows PC must be running \progname{btqw} for this to be effectual.

If the job was submitted by a user working from a client with a DHCP-allocated IP address, a message may be received on all clients
currently logged-in with that user name.

\IfXi{\subsubsection{Notes}
\progname{dosbtwrite} is identical to \OtherProductName's \progname{dosspwrite}, apart from using a different message file.}

\subsection{Jobdump}

\begin{expara}

jobdump options

\end{expara}

\progname{jobdump} is invoked by \PrBtq{}, \PrXbtq{}, \PrXmbtq{} and \PrBtjdel{} to unqueue jobs when required.

It is not intended for general use and is not documented further.

\section{File Formats}
\subsection{\masterconfigname}
\masterconfig is an optional file for reconfiguring the \ProductName{} batch management system.

This may be useful for relocating standard files and directories, such as \filename{SPOOLDIR} which defaults to \spooldir{} so that a different spool
directory is used. However completely arbitrary environment variable assignments may be made so as to pass the resulting environment on to
various subprocesses invoked by the scheduler.

Note that as the environment is assigned early within \PrBtr{}, then any jobs created will have these
environment variables assigned. However there is an alternative syntax (see below) to avoid this.

The format of the file consists of two different forms of assignment.

\begin{tabular}{l p{12cm}}
\exampletext{SPOOLDIR=/usr2/spooldir} & Sets up the given environment variable in all programs and jobs invoked by \ProductName{}.\\
& \\
\exampletext{SPOOLDIR:/usr2/spooldir} & Denotes environment variables which should not be passed on to subprocesses, or to jobs created by \PrBtr{}.\\
\end{tabular}

Comment lines may be included, introduced by a \exampletext{\#} sign, and blank lines are ignored.

The latter environment assignment format is used by default in the installation process for \ProductName{}.

Every component program of \ProductName{} examines this file and resets its environment from this file as the first step of execution.

\subsection{\hostsfilename}
\hostsfile{} is used to inform the \ProductName{} batch scheduling system, and in particular \progname{btsched} and \progname{xbnetserv}, which other host machines are to be attached.

The host machines should in general be provided for in the standard file \filename{/etc/hosts}.

The file consists of comment lines introduced by the \exampletext{\#} character, and of lines consisting of up to 4
fields, of which only the first is mandatory. These fields are as follows:

\subsubsection{Host name}
This is the name of the host as given in the \filename{/etc/hosts} file.

Alternatively an internet address of the form \exampletext{193.112.238.10} may be given if necessary and an alias is provided on the next field, but this is not recommended.

\subsubsection{Alias name}
This is the name of an alias to be used in preference to the host name to refer to the machine. To be particularly beneficial, this should be
shorter than the host name.

If this field is not required, but subsequent fields are required, then the alias name may be replaced by a single \exampletext{{}-} sign.

\subsubsection{Flags}
This is a comma-separated list of markers to denote information about the connection. The currently-supported markers are as follows:

\begin{tabular}{lp{12cm}}
\exampletext{probe} &
Indicates that a datagram should be sent, and a reply awaited,
from the host, before a full-blown connection is attempted. This is
recommended wherever possible, or it is not sure in which order
machines are booted.\\
& \\
\exampletext{manual} &
indicates that no connection at all is attempted. To connect to the machine in question, then \PrBtconn{} should
be invoked.\\
& \\
\exampletext{trusted} &
indicates that the host is ``trusted'' by the current machine, which transmits information about Windows clients and their password
validations, so the other host need not make such enquiries.\newline
N.B. This option is now deprecated.\\
& \\
\exampletext{Client (username)} &
indicates that no connection is attempted; the current machine is acting as a server for Windows clients. The specified username is to
be considered as the owner of any jobs submitted, and the user to whom charges should be applied and to which privileges apply; see
\PrBtuser{}.\newline
If \genericargs{(username)} is omitted, then the Windows user is assumed, which should correspond to a user name on the host system.\\
& \\
\exampletext{Clientuser (machine)} & Indicates that the whole entry identifies a ``roaming user'' who might be using one of
several Windows clients, possibly with IP addresses assigned via DHCP. The host name in this case is replaced by the Windows user name, and
the alias gives the Unix user name if different.\newline
If \genericargs{(machine)} is specified, then a password is demanded at the Windows client if the client's IP
address does not match that of machine.\\
& \\
\exampletext{dos(username)} &
Is a synonym for \exampletext{client(username)} kept for historical reasons.\\
& \\
\exampletext{dosuser(machine)} & Is a synonym for \exampletext{clientuser(machine)} kept for historical reasons.\\
\exampletext{external} &
Is a synonym for \exampletext{client} (no username) kept for future extensions.\\
& \\
\exampletext{pwchk} &
Always demand the user's Unix password (or a version of the password specific to \ProductName{} as set using \XipasswdName{}) when first starting up.\\
\end{tabular}
\subsubsection{Timeout}
This gives a timeout value in seconds after which the interface is to be considered closed following a connection or alternatively to await a
connection after a probe request.

A default of 1000 seconds applies if none is specified.

In the case of Windows clients, the ``login'' is considered to be dropped after this time, and the user may be asked for a password again.

\subsubsection{Local address}
On some machines, the ``local'' host address may be different from that obtained by looking at the result of \progname{gethostname}(3). To specify a different address for ``this'' machine, a line of the form:

\begin{expara}

localaddress 193.112.238.112

\end{expara}

may be specified,  but this must precede all other host names in the file.

An alternative method of getting the local address is to connect to some other server and obtain the local address by using the
\progname{getsockname} call to find the address.

To do this use the keyword \filename{GSN} thus

\begin{expara}

localaddress GSN(www.google.com,80)

\end{expara}

In this example the local address is found by attempting to connect to \filename{www.google.com} on port 80 (the \filename{http} port) and
applying \progname{getsockname} to the resulting connection.

\subsection{Static environment file}
\label{Staticenv}
Carrying all the environment around in every job can consume a lot of shared memory space and fill up the available space, often with variables
such as \filename{TERMCAP} etc which are irrelevant to batch jobs.

Common environment variables can be saved in the file \batchenv. When jobs are submitted by \PrBtr{} and similar, environment variables in the job are
compared with those specified in \batchenv{} and if they are the same, they are deleted from the environment.

A further feature introduced from \ProductName{} version \IfXi{6.437} \IfGNU{1.11} is that an empty assignment will cause the variable to be deleted.

For example \batchenv{} might contain the following:

\begin{expara}

\# Specify PATH

PATH=/bin:/usr/bin:/usr/local/bin

\# Remove irrelevant variables

TERMCAP=

LS\_COLORS=

\end{expara}

In this case \filename{PATH} would be deleted from the job environment if it contained exactly the value given and \filename{TERMCAP} and
\filename{LS\_COLORS} would be deleted from the job environment in every case.

Note that the deletions are from the environment of the job files actually created, the variables are still put into the environment of the jobs from
\batchenv{} when the jobs are actually run (with empty values for the variables such as \filename{TERMCAP} above.

The format of the file is of assignments with an \filename{=} sign, all other lines and lines starting with \filename{\#} are ignored.

Note that alternative file names to \batchenv{} or several files may be specified to provide the environment, see page \pageref{SEfiles}.

\subsection{User map file}
The user map file provides a mapping between external names, usually Windows user names, and UNIX names.

The file is in \usermap, and consists (apart from comments introduced by the \filename{\#} character) of lines of the format

\begin{expara}

unix-user:windows-user

\end{expara}

For example:

\begin{expara}

\# User mapping file

jmc:john collins

sec:sue collins

guest:default

\end{expara}

The final entry gives a default user if a named user is not found in the file.

UNIX users not found on the host are silently ignored.

The User Mapping file is used to convert Windows user names to UNIX user names for the Windows clients, or for the API functions.

