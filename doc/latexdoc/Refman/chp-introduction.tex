\chapter{Introduction}
\label{chp:introduction}
\ProductName{} is a fully functioned, high performance Job Scheduler and
Management System which is available for a wide range of machines
running a Unix Operating System. This manual provides the System
Reference Information for all of the Unix platforms on which \ProductName{}
may be run, covering the basic product, shell and ``curses'' interfaces and the Motif
Interface.

Separate manuals discuss the MS Windows Clients, the Web Browser Interface and the API.

\section{Typographical Conventions}
\IfXi{Xi Software}\IfGNU{These} manuals use various character fonts to indicate different
types of information as follows:

\begin{center}
\filename{File names and quotations within the text}

\exampletext{Examples and user script}

\genericargs{Generic data (where you should put a value appropriate to your own environment)}

\progname{Program names, whether for \ProductName{} or standard Unix facilities}

\warnings{Warnings and important advice}

\end{center}

\section{Command Line Program Options}
Almost all of the programs that make up \ProductName{} can take (or require)
options and arguments supplied on the command line. As much flexibility
as possible is allowed in the specification of these options and
arguments. The examples in the manual use which ever notation is
clearest.

White space may be inserted into flag arguments as in

\begin{expara}

\BtrName{} -c COUNT=0 -T 10:16

\end{expara}

or it may be left out as in

\begin{expara}

\BtrName{} -cCOUNT=0 -T10:16

\end{expara}

Single character options may be strung together with one minus sign:

\begin{expara}

\BtrName{} -mwC

\end{expara}

or separated, as in

\begin{expara}

\BtrName{} -m -w -C

\end{expara}

If mutually contradictory arguments are permitted, the rightmost (or rather the most recently specified) applies.

The ability to redefine option letters has been provided, together with the \exampletext{+keyword} or
\exampletext{{}-{}-keyword} style of option. Such options should be given completely surrounded by spaces or tabs to separate
them from each other and their arguments, for example

\begin{expara}

\BtrName{} -{}-condition COUNT=0 -{}-time 10:16

\end{expara}

In addition, all the commands have an option \exampletext{{}-?} or \exampletext{\textit{+explain}} (or \exampletext{{}-{}-explain}) whose function is to list all the other options and exit.

There is a mechanism for picking up options from environment variables or so-called configuration files called \configurationfile{}
or \homeconfigpath{} off the user's home directory containing the relevant keyword.

