\subsection{\BtstartName}

\begin{expara}

\BtstartName{} [-options]

\end{expara}

\PrBtstart{} initiates the \ProductName{} batch scheduler system, by starting the processes \progname{btsched} and
\progname{xbnetserv}.

\subsubsection[Options]{Options}
The environment variable on which options are supplied is \filename{\BtstartVarname} and the environment variable to specify the
help file is \filename{BTRESTCONF}.

\PrBtstart{} does not do anything, and most of the options are obviously ignored, if the scheduler is already running.

\setbkmkprefix{btstart}
\explainopt

\cmdoption{l}{initial-load-level}{number}{initll}

This option arranges for the \filename{LOADLEVEL} variable, which controls the total load level of running jobs to the
specified number (usually zero).

This is useful for starting up in a controlled fashion, checking the status of jobs and then resetting \filename{LOADLEVEL} appropriately allowing jobs to run.

If this option is not specified, then the value is unchanged from its initial value saved by the scheduler when it was last shut down.

\cmdoption{j}{initial-job-size}{number}{initjobs}

Allocate space for the the specified maximum number of jobs on startup.

This is very often necessary as it may not be possible to reallocate additional shared memory after processing has got under way.

If this option is not specified, then the initial allocation will be based on the original number of saved jobs, which is often far from enough.

\cmdoption{v}{initial-var-size}{number}{initvars}

Allocate space for the specified maximum number of variables on startup.

This is very often necessary as it may not be possible to reallocate additional shared memory after processing has got under way.

If this option is not specified, then the initial allocation will be based on the original number of saved variables, which is often far from enough.

\freezeopts{\filename{\BtstartVarname}}{Stop}

