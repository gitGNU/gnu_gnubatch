\subsection{\BtqName}

\begin{expara}

\BtqName{} [ -options ]

\end{expara}

\PrBtq{} is an interactive program that allows the user to display in real-time state and details of
\ProductName{} jobs and variables on local or remote machines, refreshing the screen automatically as the queue
changes or variables are updated, and allowing the status of jobs and variables on the queue to be altered according to each
user's permissions and privileges.

Please see page \pageref{bkm:Btqdescr} for more details of the interactive commands.
This section focuses on the command-line options which may be used to control the initial display.

\subsubsection{Options}
The environment variable on which options are supplied is \filename{\BtqVarname} and the environment variable to specify the
help file is \filename{BTQCONF}.

Certain commands available on-screen enable many of these options to be changed and saved in configuration files.

\setbkmkprefix{btq}
\explainopt

\cmdoption{A}{no-confirm-delete}{}{noconfdel}

Suppress confirmation request for delete operations.

\cmdoption{a}{confirm-delete}{}{confdel}

Ask confirmation of delete operations (this is the default).

\cmdoption{B}{no-help-box}{}{nohelpbox}

Put help messages in inverse video rather than in a box (this is the default).

\cmdoption{b}{help-box}{}{helpbox}

Put help messages in a box rather than displaying inverse video.

\cmdoption{E}{no-error-box}{}{noerrorbox}

Put error messages in inverse video rather than in a box.

\cmdoption{e}{error-box}{}{errorbox}

Put error messages in a box rather than displaying inverse video.

\cmdoption{g}{just-group}{group}{group}

Restrict the output to jobs owned by the group specified.
Cancel this argument by giving a single \exampletext{{}-} sign
as an argument. The group name may be a pattern with shell-like wild cards.

\cmdoption{H}{keep-char-help}{}{keepchhelp}

When displaying a help screen, interpret the next key press as a command as well as clearing the help screen. This is the
default.

\cmdoption{h}{lose-char-help}{}{losechhelp}

Discard whatever key press is made to clear a help screen.

\cmdoption{j}{jobs-screen}{}{jobscreen}

Commence display in jobs screen. This is the default unless there are no jobs to be viewed.

\cmdoption{l}{local-only}{}{localonly}

Only display jobs or variables local to the machine.

\cmdoption{N}{follow-job}{}{followjob}

If the currently selected job or variable moves on the screen, try to follow it and do not try to retain relative screen positions.

\cmdoption{q}{job-queue}{name}{queue}

Restricts attention to jobs with the queue prefix \genericargs{name}.

Cancel this argument by giving a single \exampletext{{}-} sign as an argument. The queue name may be a pattern with shell-like wild
cards.

\cmdoption{r}{network-wide}{}{networkwide}

Display jobs on all connected hosts.

\cmdoption{s}{keep-cursor}{}{keepcursor}

If the currently-selected job or variable moves on the screen, try to preserve the
relative position of the cursor on the screen rather than following the
job or variable.

\cmdoption{u}{just-user}{user}{user}

Restrict the output to jobs owned by the user specified. Cancel this argument by giving a \exampletext{{}-} parameter.

The user name may be a pattern with shell-like wild cards.

\cmdoption{v}{vars-screen}{}{varsscreen}

Commence display in variables screen.

\cmdoption{Z}{no-null-queues}{}{nonull}

In conjunction with the \exampletext{{}-q} parameter, do not include jobs with no queue prefix in the list.

\cmdoption{z}{null-queues}{}{null}

In conjunction with the \exampletext{{}-q} parameter, include jobs with no queue prefix in the list.

