\subsection{Bthols}

\begin{expara}

\BtholsName{} [-C] [-d] [-r] [-s] year [file]

\end{expara}

\PrBthols{} is a shell-level program to display or set the holidays file for the given year.

The holidays are displayed or interpreted in the following format (as for UK in 2004)

\begin{expara}

January: 1

April: 9 12

May; 3 31

August: 30

December: 27 28

\end{expara}

The year is given as 4 digits, thus \exampletext{2014}. Output when displaying, the
default, is to standard output.

If setting (with the \exampletext{{}-s} option) the input is from standard input or the specified file name. Holidays are added to
the existing list for the year unless the \exampletext{{}-C} option is also given.

Month names may be given in abbreviated or full format, case-insensitive, but are displayed in full. The full and abbreviated
names are extracted from the help file, by default \filename{\helpdirname/btrest.help}.

\subsubsection{Options}
The environment variable on which options are supplied is \filename{\BtholsVarname} and the environment variable to specify the
help file is \filename{BTRESTCONF}.

\setbkmkprefix{bthols}
\explainopt

\cmdoption{C}{clear}{}{clear}

Relevant only when the \exampletext{{}-s} option is specified, clear the existing holidays for the given year before
applying the new ones.

\cmdoption{d}{display}{}{display}

Display the existing holidays.

This is the default if no options are given.

\cmdoption{s}{set}{}{set}

Set holidays from standard input.

\cmdoption{r}{no-clear}{}{noclear}

Cancel any previously-set \exampletext{{}-C} option.

\freezeopts{\filename{\BtholsVarname}}{STOP}

