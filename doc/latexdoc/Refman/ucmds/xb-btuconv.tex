\subsection{\XbBtuconvName}

\begin{expara}

\XbBtuconvName{} [-D dir] [-v n] [-e n] [-s] [-f] user file outfile

\end{expara}

\PrXbBtuconv{} converts the \ProductName{} user file, which is usually
\filename{btufile}, possibly with a numeric suffix, held in the batch spool directory to an executable shell script file
\genericargs{outfile}, which if executed, would recreate the \ProductName{}
users' permissions with the same options and privileges.

\IfXi{\PrXbBtuconv{} understands the format of the saved user file for versions of \ProductName{} going back
to release 4, and when presented with a saved file, will attempt to work out from the format which release it relates to.}

In addition to options, two arguments are always supplied to \PrXbBtuconv{}.

\begin{tabular}{l p{14cm}}
User file & This is the file containing the attributes of the user permissions,
\filename{btufile\textit{n}} in the batch spool directory, by default \spooldir, or as relocated by re-specifying
\filename{SPOOLDIR}.\\
& \\
Output file & This file is created by \PrXbBtuconv{} to contain the executable shell script, containing
\PrBtuchange{} commands, which may be used to recreate the user file.
This file should be run before restarting the scheduler.\\
\end{tabular}

\subsubsection{Options}

Note that this program does not provide for saving options in \configurationfile{} or \homeconfigpath{} files.

\setbkmkprefix{xb-btuconv}

\cmdoption{D}{}{directory}{directory}

This option specifies the source directory for the users and user file. It can be specified as \exampletext{\$SPOOLDIR} or
\exampletext{\$\{SPOOLDIR-/var/spool/batch\}} etc and the environment and/or \linebreak[3]\masterconfig{} will be
interrogated to interpolate the value of the environment variable given.

If you use this, don't forget to put single quotes around it thus:

\begin{expara}

\XbBtuconvName{} -D {\textquotesingle}\$\{SPOOLDIR-\spooldirname\}{\textquotesingle} ...

\end{expara}

otherwise the shell will try to interpret the \exampletext{\$} construct and not \PrXbBtuconv.

\cmdoption{e}{}{n}{errorlim}

Tolerate \genericargs{n} errors of the kinds denoted by the other options before giving up trying to convert the file.

\cmdoption{f}{}{}{ignfmt}

Ignore format errors in the saved user file, up to the limit of errors given by the \exampletext{{}-e} option.

\cmdoption{s}{}{}{ignsize}

Ignore file size errors in the saved user file (up to the number of total errors given by the \exampletext{{}-e}
option.

\IfXi{\cmdoption{v}{}{number}{version}

Tell \PrXbBtuconv{} that the user file is for release number of \ProductName{}, where number
is 4 to 6. This may be necessary where the user file is corrupted and
\PrXbBtuconv{} cannot guess what is meant.}

