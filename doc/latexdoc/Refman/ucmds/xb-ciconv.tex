\subsection{\XbCiconvName}

\begin{expara}

\XbCiconvName{} [-D dir] [-v n] [-e n] [-u] [-s] [-f] cifile outfile

\end{expara}

\PrXbCiconv{} converts \ProductName{} command interpreters held in the
batch spool directory to an executable shell script which may be used
to re-install them. This may be useful for backup purposes or for one
stage in upgrade from one release of \ProductName{} to another.

\PrXbCiconv{} understands the format of the saved job file for versions of \ProductName{} going back to
release 4, and when presented with a saved file, will attempt to work out from the format which release it relates to.

In addition to options, two arguments are always supplied to \PrXbCiconv{}.

\begin{tabular}{l p{12cm}}
Command interpreter list file &
This is the file containing the attributes of the variables, \filename{cifile} in the batch spool directory, by default
\spooldir, or as relocated by re-specifying \filename{SPOOLDIR}.\\
& \\
Output file & This file is created by \PrXbCiconv{} to contain the executable shell script, containing
\PrBtcichange{} commands, which may be used to recreate the command interpreters.\\
\end{tabular}

\subsubsection{Options}

Note that this program does not provide for saving options in \configurationfile{} or \homeconfigpath{} files.

\setbkmkprefix{xb-ciconv}

\cmdoption{D}{}{directory}{directory}

This option specifies the source directory for the variables rather than the current directory. It can be specified as
\exampletext{\$SPOOLDIR} or
\exampletext{\$\{SPOOLDIR-\spooldirname\}} etc and the environment and/or \linebreak[3]\masterconfig{} will be
interrogated to interpolate the value of the environment variable given.

If you use this, don't forget to put single quotes around it thus:

\begin{expara}

\XbCiconvName{} -D
{\textquotesingle}\$\{SPOOLDIR-/usr/spool/batch\}{\textquotesingle}
...

\end{expara}

otherwise the shell will try to interpret the \exampletext{\$} construct and not \PrXbCiconv{}.

\cmdoption{e}{}{n}{errorlim}

Tolerate \genericargs{n} errors of the kinds denoted by the other options before giving up trying to convert the file.

\cmdoption{f}{}{}{ignfmt}

Ignore format errors in the saved variables file, up to the limit of errors given by the \exampletext{{}-e} option.

\cmdoption{s}{}{}{ignsize}

Ignore file size errors in the saved variables file (up to the number of total errors given by the \exampletext{{}-e}
option.

\IfXi{\cmdoption{v}{}{number}{version}

Tell \PrXbCiconv{} that the variables file is for release number of \ProductName{}, where number
is 4 to 6. This may be necessary where the user file is corrupted and \PrXbCiconv{} cannot guess what is meant.}

