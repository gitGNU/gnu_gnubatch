\subsection{\BtulistName}

\begin{expara}

\BtulistName{} [-options] [user ...]

\end{expara}

\PrBtulist{} lists the permissions of users known to the \ProductName{} batch scheduler system. All users are listed
if no users are specified, otherwise the named users are listed. The report is similar to the main display of
\PrBtuser{}.

The invoking user must have \textit{read admin file} permission to use \PrBtulist{}.

\subsubsection{Options}
The environment variable on which options are supplied is \filename{\BtulistVarname} and the environment variable to specify the
help file is \filename{BTRESTCONF}.

\setbkmkprefix{btulist}
\explainopt

\cmdoption{D}{default-format}{}{deffmt}

Cancel the \exampletext{{}-F} option and revert to the default format.

\cmdoption{d}{default-line}{}{defline}

Display an initial line giving the default options (included by default).

\cmdoption{F}{format}{format}{format}

Format the output according to the format string given.

\cmdoption{g}{group-name-sort}{}{grpsort}

Sort the list of users by the group name in ascending order, then by users within that group as primary group.

\cmdoption{H}{header}{}{header}

Generate a header for each column of the output.

\cmdoption{N}{no-header}{}{nohdr}

Cancel the \exampletext{{}-H} option.

\cmdoption{n}{numeric-user-sort}{}{numusort}

Sort the list of users by the numeric user id (default).

\cmdoption{S}{no-user-lines}{}{nousers}

Suppress the user lines. It is an error to invoke this and the \exampletext{{}-s} option as well.

\cmdoption{s}{no-default-line}{}{nodef}

Suppress the initial line giving the default options. It is an error to invoke this and the \exampletext{{}-S} option as well.

\cmdoption{U}{user-lines}{}{ulines}

Display the user lines (default).

\cmdoption{u}{user-name-sort}{}{usort}

Sort the list of users by the user name.

\freezeopts{\filename{\BtulistVarname}}{}

\subsubsection{Format argument}

The format string consists of a string containing the following character sequences, which are replaced by various user permission
parameters. The string may contain various other printing characters or spaces as required.

Each column is padded on the right to the length of the longest entry.

If a header is requested, the appropriate abbreviation is obtained from
the message file and inserted.

\begin{center}
\begin{tabular}{|l|l|}
\hline
\exampletext{\%\%} & Insert a single \exampletext{\%} character.\\\hline
\exampletext{\%d} & Default priority\\\hline
\exampletext{\%g} & Group name\\\hline
\exampletext{\%j} & Job mode\\\hline
\exampletext{\%l} & Minimum priority\\\hline
\exampletext{\%m} & Maximum priority\\\hline
\exampletext{\%p} & Privileges\\\hline
\exampletext{\%s} & Special create load level\\\hline
\exampletext{\%t} & Total load level\\\hline
\exampletext{\%u} & User name.\\\hline
\exampletext{\%v} & Variable mode\\\hline
\exampletext{\%x} & Maximum load level\\\hline
\end{tabular}
\end{center}

The string \exampletext{DEFAULT} replaces the user name in the default values line, or the group name if the user name is not printed.
If the group name is not printed as well, then this will be omitted and will be indistinguishable from the rest of the output.

Note that the various strings are read from the message file, so it is possible to modify them as required by the user.

The default format is

\begin{expara}

\%u \%g \%d \%l \%m \%x \%t \%s \%p

\end{expara}

\subsubsection{Privileges format}
The following are output via the \%p format. Note that the actual strings are read from the message file, and are the same ones as are
used by \PrBtuchange.

\begin{center}
\begin{tabular}{|l|l|}
\hline
\exampletext{RA} & read admin file\\\hline
\exampletext{WA} & write admin file\\\hline
\exampletext{CR} & create\\\hline
\exampletext{SPC} & special create\\\hline
\exampletext{ST} & stop scheduler\\\hline
\exampletext{Cdft} & change default\\\hline
\exampletext{UG} & or user and group modes\\\hline
\exampletext{UO} & or user and other modes\\\hline
\exampletext{GO} & or group and other modes.\\\hline
\end{tabular}
\end{center}

\exampletext{ALL} is printed if all privileges are set.

\subsubsection{Modes}
Modes printed by the \%j and \%v options are as follows:

\begin{center}
\begin{tabular}{|l|l|}
\hline
\exampletext{R} & read permission\\\hline
\exampletext{W} & write permission\\\hline
\exampletext{S} & reveal permission\\\hline
\exampletext{M} & read mode\\\hline
\exampletext{P} & set mode\\\hline
\exampletext{U} & give away owner\\\hline
\exampletext{V} & assume owner\\\hline
\exampletext{G} & give away group\\\hline
\exampletext{H} & assume group\\\hline
\exampletext{D} & delete\\\hline
\exampletext{K} & kill (only valid for jobs)\\\hline
\end{tabular}
\end{center}

Each section of the mode (job, group, others) is represented by the prefixes \exampletext{U:}, \exampletext{G:} and
\exampletext{O:} and separated by commas.

For example:

\begin{expara}

U:RWSMPDK,G:RWSDK,O:RS

\end{expara}

This is exactly the same format as is expected by \PrBtuchange{} etc.

