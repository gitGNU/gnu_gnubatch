\subsection{\BtuserName}
\setbkmkprefix{btuser}

\begin{expara}

\BtuserName{} [ -options ]

\end{expara}

\PrBtuser{} provides 4 functions:

With no arguments it lists the permissions for the invoking user and exits.

\bookmark{permsreq}
With the \exampletext{{}-m} option it enables the invoking user to edit his own default job and variable permissions. The user must
have change default modes permission to do this.

With the \exampletext{{}-v} option it enables the invoking user to interactively review the list of permissions for all users. The user
must have read admin file permission to do this.

With the \exampletext{{}-i} option it enables the invoking user to interactively review and update the list of permissions for all
users. The user must have write admin file permission to do this.

Please see page \pageref{bkm:Btuserdescr} for more details of the interactive commands.
This section focuses on the command-line options which may be used to control the initial display.

\subsubsection{Options}
The environment variable on which options are supplied is \filename{\BtuserVarname} and the environment variable to specify the
help file is \filename{BTUSERCONF}.

Certain commands available on-screen enable many of these options to be changed and saved in configuration files.

\explainopt

\cmdoption{B}{no-help-box}{}{nohelpbox}

Put help messages in inverse video rather than in a box (this is the default).

\cmdoption{b}{help-box}{}{helpbox}

Put help messages in a box rather than displaying inverse video.

\cmdoption{d}{display-only}{}{disponly}

This is the default. A list of permissions is output to the standard output.

\cmdoption{E}{no-error-box}{}{noerrorbox}

Put error messages in inverse video rather than in a box.

\cmdoption{e}{error-box}{}{errorbox}

Put error messages in a box rather than displaying inverse video.

\cmdoption{g}{group-name-sort}{}{grpsort}

Sort the list of users by the group name in ascending order, then by users within that group as primary group.

This is only relevant to \exampletext{{}-v} or \exampletext{{}-i} options.

\cmdoption{H}{keep-char-help}{}{keepchhelp}

When displaying a help screen, interpret the next key press as a command as well as clearing the help screen. This is the
default.

\cmdoption{h}{lose-char-help}{}{losechhelp}

Discard whatever key press is made to clear a help screen.

\cmdoption{i}{update-users}{}{updusers}

View and update the list of users. This option requires \textit{write admin file} privilege.

\cmdoption{m}{set-default-modes}{}{setdef}

Interactively set the default modes for the invoking user. This option requires \textit{change default modes} privilege.

\cmdoption{n}{numeric-user-sort}{}{numusort}

Sort the list of users by the numeric user id (default).

\cmdoption{u}{user-name-sort}{}{usort}

Sort the list of users by the user name.

\cmdoption{v}{view-users}{}{viewusers}

View in read-only mode the list of users and permissions. This requires \textit{read admin file} privilege.

