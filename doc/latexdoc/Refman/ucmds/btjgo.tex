\subsection{\BtjgoName, \BtjgoadvName, \BtjadvName}

\begin{expara}

\BtjgoName{} job number ...

\BtjgoadvName{} job number ...

\BtjadvName{} job number ...

\end{expara}

\PrBtjgo{} forces a job or jobs to run, ignoring the ``next run time''. Conditions and load level
constraints are however still enforced. The ``next run time'' will not be affected when the job completes. This
inserts an extra run of the job.

\PrBtjgoadv{} forces a job or jobs to run, ignoring the ``next run time''. Conditions and load
level constraints are however still enforced. The ``next run time'' is advanced to the next time. This brings
forward the next run, thereafter resuming the sequence.

\PrBtjadv{} advances the run time on each job specified to the next run time according to its repeat time without
running the job or looking at conditions.

These programs are all links to the \PrBtjdel{} binary. They all parse the same options as \PrBtjdel, but do not do anything with them.

Jobs are specified by using the job number, as displayed by \PrBtr{} with the \exampletext{{}-v}
(verbose) option, or as in the output of the first column of the \PrBtjlist{} command with default format.

Remote jobs should be specified by prefixing the job numbers with the host name thus:

\begin{expara}

host:1234

\end{expara}

It is not necessary to specify any leading zeroes.

