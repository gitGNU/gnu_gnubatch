\subsection{\XipasswdName}

\begin{expara}

\XipasswdName{} [-u user] [-p password] [-f] [-d] [-F file]

\end{expara}

\PrXipasswd{} sets a password for the current user or a specified user if \exampletext{{}-u} is given. This is
separate and distinct from the user's login password.
This password is used by the web interfaces, the Windows interfaces and
the APIs\IfXi{ for both \ProductName{} and \OtherProductName}.

The reason for doing this is because it is considered insecure to possibly repeatedly try login passwords from user programs.

If any users have a password set in this way, then all users will have to to have a password in the file to use any of the interfaces
requiring a password.

Unlike the Unix \progname{passwd(1)} routine, the old password is not prompted for and there is no confirmation.

\subsubsection{Options}

Note that this program does not provide for saving options in \configurationfile{} or \homeconfigpath{} files.

\setbkmkprefix{xipasswd}

\cmdoption{u}{}{user}{user}

Set password for given user. This may only be for other than the current user if \PrXipasswd{} is invoked by
\filename{root}.

\cmdoption{p}{}{passwd}{passwd}

Specify the password to be set other than prompting for it.

\cmdoption{f}{}{}{insist}

It is normally considered an error to include a password for \filename{root} for the same reasons that the password file is
separate. However this option may be set to insist upon it.

\cmdoption{d}{}{}{deluser}

Delete the user's password from the file.

\cmdoption{F}{}{file}{file}

Use \genericargs{file} to hold the password. The default if no file is given
is \filename{\IfXi{/usr/spool/progs/xipwfile}\IfGNU{/usr/local/share/gbpwfile}}. Any number of
\exampletext{{}-F} options may be given to set up several password files at once.

