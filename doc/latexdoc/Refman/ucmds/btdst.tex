\subsection{\BtdstName}

\begin{expara}

\BtdstName{} [ -R ] startdate enddate adjustment

\end{expara}

\PrBtdst{} adjusts all jobs between the specified start and end dates and times by adding the specified (possibly signed)
adjustment in seconds to it.

The dates and times may be specified in the forms

\begin{expara}

dd/mm

mm/dd

yy/mm/dd

\end{expara}

Which of the first two forms is chosen is taken from the existing time zone. For time zones greater or equal to 4 West from GMT, the
\genericargs{mm/dd} form is chosen, otherwise \genericargs{dd/mm}.

The dates may be followed by a comma and a time in the form \genericargs{hh:mm}, otherwise midnight is assumed.

When working out what to do, remember that Unix internal time is based upon Greenwich Mean Time (GMT), it is the display which changes, so
that the effect of moving the clocks forward is to make the times (held as GMT) appear later than they did before.

A negative adjustment is subtracted from the time, making jobs run sooner. This is therefore appropriate when the clocks go forward at the
start of the summer time. Likewise a positive adjustment should be used at the end of summer time.

The optional argument \exampletext{{}-R} tries to apply the option to all exported remote jobs, but this really is not recommended
as the local jobs on those hosts will be unaffected probably leaving the users on those machines confused.

