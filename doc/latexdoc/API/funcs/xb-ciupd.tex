\subsection{\funcnameXBciupd{}}

\begin{expara}

int \funcnameXBciupd{}(const int fd,

\ \ \ \ \ \ \ \ \ \ \ \ \ const unsigned flags,

\ \ \ \ \ \ \ \ \ \ \ \ \ const int indx,

\ \ \ \ \ \ \ \ \ \ \ \ \ const Cmdint *newci)

\end{expara}

The function \funcXBciupd{} is used to update the
details of a command interpreter on a
\ProductName{} server. The invoking user must have
\textit{special create} permission or the call will be rejected.

\exampletext{fd} is a file descriptor which was previously
returned by a successful call to \funcXBopen{} or equivalent.

\exampletext{flags} is currently unused, but is reserved for
future extensions. Set it to zero.

\exampletext{indx} is the number of the command interpreter to
be updated (see \funcXBciread{}).

\exampletext{newci} is a pointer to a structure containing the
new command interpreter details.

The definition of the command interpreter structure is given on page
\pageref{bkm:Cmdinterp} onwards.

\subsubsection{Return values}
The function returns 0 if successful otherwise one of the error codes
listed on page \pageref{errorcodes} onwards.

