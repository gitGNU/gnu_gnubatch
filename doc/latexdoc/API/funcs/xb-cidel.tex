\subsection{\funcnameXBcidel{}}

\begin{expara}

int \funcnameXBcidel{}(const int fd, const unsigned flags, const unsigned indx)

\end{expara}

The function \funcXBcidel{} is used to delete a
command interpreter from a \ProductName{} server.
The invoking user must have \textit{special create} permission or the
call will be rejected.

\exampletext{fd} is a file descriptor which was previously
returned by a successful call to \funcXBopen{} or equivalent.

\exampletext{flags} is currently unused, but is reserved for
future extensions. Set it to zero.

\exampletext{indx} is the number of the command interpreter to
be deleted.

\subsubsection{Return values}
The function returns 0 if successful otherwise one of the error codes
listed on page \pageref{errorcodes} onwards.

\subsubsection{Notes}
The standard shell entry, entry zero, cannot be deleted and attempts to
do so will always return an error code
(\filename{\constprefix{}BAD\_CI}).

There are few checks and interlocks on command interpreter lists, which
are assumed to be likely to be changed sparingly. The user should
satisfy him or herself that there are no jobs likely to use the command
interpreter about to be deleted before invoking this operation.

