\chapter{Introduction}
\label{chp:introduction}
\ProductName{} is a fully functioned, high performance Job Scheduler and
Management System which is available for a wide range of machines
running a Unix-style Operating System.

\IfGNU{The original version was written by John Collins at Xi Software Ltd and marketed as
\strongemphasis{Xi-Batch} (this name may still appear in some diagrams). The names, including many
of the program and interface names, have been changed to \ProductName{} or derivatives
and the default installation directories have been changed to GNU standards.}

The product consists of a ``core product'' or ``basic product'' which contains the
scheduling software, command-line and character-based interfaces.
Additional options provide for:

\begin{itemize}
\item An X-Windows GTK+ or Motif Toolkit Interface. \IfGNU{This is not really supported under GNU but the programs do build and run.}
\item An API for use with C and C++ \IfGNU{This has been adapted to work under Java.}
\IfXi{\item An Interface for MS-Windows using Visual C++}
\item An Interface for MS-Windows using Python and PyQt (this can also be used on other platforms supporting PyQt\IfGNU{, including GNU/Linux,} as an alternative interface).
\item A C and C++ API for use with MS-Windows
\item Browser Interfaces
\end{itemize}
The basic manuals cover the ``basic product'' and the X-Windows interfaces. Additional supplements cover the other
interfaces.

The basic manuals are:

\begin{itemize}
\item User Guide - a quick introduction and ``cookbook'' for use of \ProductName{}
\item Reference manual - a complete description of all components of the basic product.
\item Administrator Guide - this manual. Information about installation and customisation of the software.
\end{itemize}
Also available are:

\begin{itemize}
\item API reference manual for Unix and MS-Windows API
\IfXi{\item MS Windows Interface Manual}
\item MS Windows PyQt Interface Manual
\item Browser Interface Manual
\end{itemize}
\section{Documentation Standards}
\IfXi{Xi Software}\IfGNU{These} manuals use various character fonts to indicate different types of information as follows:

\begin{center}
\filename{File names and quotations within the text}

\exampletext{Examples and user script}

\genericargs{Generic data (where you should put a value appropriate to your own environment)}

\userentry{Something you should type}

\progname{Program names, whether for \ProductName{} or standard Unix facilities}

\warnings{Warnings and important advice}
\end{center}

\section{Command Line Program Options}
Almost all of the programs that make up \ProductName{} can take (or require)
options and arguments supplied on the command line. As much flexibility
as possible is allowed in the specification of these options and
arguments. The examples in the manual use which ever notation is
clearest.

White space may be inserted into flag arguments as in

\begin{expara}

\BtrName{} -c COUNT=0 -T 10:16

\end{expara}

or it may be left out as in

\begin{expara}

\BtrName{} -cCOUNT=0 -T10:16

\end{expara}

Single character options may be strung together with one minus sign:

\begin{expara}

\BtrName{} -mwC

\end{expara}

or separated, as in

\begin{expara}

\BtrName{} -m -w -C

\end{expara}

If mutually contradictory arguments are permitted, the rightmost (or
rather the most recently specified) applies.

The ability to redefine option letters has been provided, together with
the \exampletext{+keyword} or \exampletext{{}-{}-keyword} style of option. Such options
should be given completely surrounded by spaces or tabs to separate
them from each other and their arguments, for example

\begin{expara}

\BtrName{} +condition COUNT=0 -{}-time 10:16

\end{expara}

In addition, all the commands have an option \exampletext{{}-?}
or \exampletext{+explain} (or \exampletext{{}-{}-explain}) whose function is to list all the
other options and exit.

